%------------------------------------------------------------------------------------------------------------------------
\chapter{次世代ピクセルモジュールの量産}
\label{sec:singatapixel-devel}
%------------------------------------------------------------------------------------------------------------------------
RUN3に向けた現行ピクセルモジュールの測定準備に加え、ATLASではHL-LHCアップグレードに向けた内部飛跡検出器の総入れ替えのため、次世代ピクセルモジュールの開発が進められている。現在、ITkに搭載するピクセルモジュール量産の各組み立て工程における試験やそのシステム確立のため、試作器を用いたデモンストレーションが行われている。

日本では次世代器量産の際に約$2000$個のモジュールを生産する予定である。次世代ピクセルモジュールの量産の際に、効率の良い量産と統合されたピクセルモジュール選定を行うため、品質試験結果を統合管理するシステムの開発が必要となる。

本章において、ITkに搭載する次世代ピクセルモジュールの生産の流れおよび品質試験の概要を説明し、\ref{sec:chap6}章および\ref{sec:chap7}章において本研究における開発項目である品質試験データ管理システムについて説明する。

%------------------------------------------------------------------------------------------------------------------------
\section{次世代ピクセルモジュールの組み立て部品}
\label{sec:component}
%------------------------------------------------------------------------------------------------------------------------

ITkに搭載するピクセルモジュールは以下の3種類を予定している。
\begin{itemize}
  \item ステーブ用Tripletモジュール
  \item リング用Tripletモジュール
  \item Quadモジュール
\end{itemize}

\fref{fig:itkpixelhaiti}にそれぞれの種類のピクセルモジュールの配置を示す。
Tripletモジュールはシリコンセンサー1つ、FE チップ1枚から構成されるSingleベアモジュールを3つ搭載する。一方で、Quadモジュールはシリコンセンサー1つ、FE チップ4枚から構成されるQuadベアモジュールを1つ搭載する。先述のように日本では約2000個のモジュールを生産する予定であるが、これは全てQuadモジュールである。


\begin{figure}[tbp]
  \centering
  \includegraphics[height=8cm,keepaspectratio]{pixeltripletquad.png}
  \caption[TripletモジュールおよびQuadモジュールの配置]{TripletモジュールおよびQuadモジュールの配置\ \cite{itkpixellayoutupdate}。原点はITkの中心であり、$z$軸はビーム軸方向、$r$はビーム軸と垂直な半径を表す。最内層はTripletモジュールであり、$|z|<261\ \si{mm}$である図中の紫色の四角内にステーブ用Tripletモジュール (合計288個)、$|z|>261\ \si{mm}$である図中の青色の四角内はリング用Tripletモジュール (合計900個)を搭載する。最内層以外はQuadモジュール (合計7976個)を搭載する。}
  \label{fig:itkpixelhaiti}
\end{figure}


%\begin{table}[tbp]
%  \begin{center}
%    \caption[ITkに搭載するピクセルモジュールの配置]{ITkに搭載するピクセルモジュールの配置。ピクセルモジュールはITkの内側五層に配置され、それぞれの層を内側から順にL0、L1、L2、L3、L4と呼ぶ。}
%    \label{tab:itkpixelhaiti}
%    \scalebox{0.8}{
%    \begin{tabular}{|l|c|c|c|c|c|}
%    \hline
%      Layer & センサータイプ & センサー厚み[$\si{\micro m^2}$] & センサーサイズ [\si{\micro m}] & モジュールタイプ & モジュール数 \\
%    \bhline{1.5pt}
%     L0 (バレル部) & 3D n-in-p & 150 & $25\times100$ & Triplet & 288 \\
%    \hline
%    L0 (エンドキャップ部) & 3D n-in-p & 150 & $50\times50$ & Triplet & 900  \\
%    \hline
%    L1& プラナー n-in-p & 100 & $50\times50$ & Quad & 1160  \\
%    \hline
%    L2-4 & プラナー n-in-p & 100 & $50\times50$ & Quad & 6816  \\
%    \hline
%    \end{tabular}
%    }
%  \end{center}
%\end{table}


ピクセルモジュールの模式図を\fref{fig:quadkousei}に示す。量産工程は、各組み立て機関に届いたセンサーとFEチップから作られるベアモジュールとフレキシブル基板の接着から始まる。本節では各部品の詳細について説明する。


\begin{figure}[tbp]
  \centering
  \includegraphics[height=6cm,keepaspectratio]{quadkousei.png}
  \caption[ピクセルモジュールの模式図]{ピクセルモジュールの模式図。この図はQuadモジュールの構成を表したものである。ピクセルモジュールはベアモジュールとフレキシブル基板を貼り付けることで作られる。}
  \label{fig:quadkousei}
\end{figure}

%------------------------------------------------------------------------------------------------------------------------
\subsection{ベアモジュール}
\label{sec:bare}
%------------------------------------------------------------------------------------------------------------------------

ベアモジュールはセンサーとFEチップをバンプ接合することにより作られる。Quadモジュールではセンサー1枚に対してFEチップ4枚、Tripletモジュールではセンサー1枚に対してFEチップ1枚から構成される。ベアモジュールは通過する粒子を検出する。センサーを通過した荷電粒子は電子・ホール対を生成し、それにより得られる信号をFEチップを用いて増幅・整形を行う。その後、デジタル処理を行いToTを出力する。

%\fref{fig:bare}にベアモジュールの全体図を示す。
%\begin{figure}[tbp]
%  \centering
%  \includegraphics[height=5cm,keepaspectratio]{bare.png}
%  \caption[ベアモジュール]{ベアモジュールの全体図。センサー側から見たものであり、左右にFEチップがはみ出している。これはフレキシブル基板につながるワイヤーのためのパッド部分である。}
%  \label{fig:bare}
%\end{figure}

%現在行われている試作器の組み立てではRD53AというFEチップを用いている。


%------------------------------------------------------------------------------------------------------------------------
\subsection{フレキシブル基板}
\label{sec:flex}
%------------------------------------------------------------------------------------------------------------------------

フレキシブル基板はセンサーの裏側に接着、およびワイヤー配線によりFEチップと電気的に接続される。フレキシブル基板の全体図を\fref{fig:flex}に示す。フレキシブル基板は、以下の3つの役割を持つ。

\begin{figure}[tbp]
  \centering
  \includegraphics[height=6cm,keepaspectratio]{flex.png}
  \caption[フレキシブル基板]{フレキシブル基板の全体図\ \cite{itk}。}
  \label{fig:flex}
\end{figure}

\begin{itemize}
  \item FEチップからの信号輸送  \\
  センサーから得られた信号はFEチップで増幅・整形およびデジタル処理が行われ、フレキシブル基板に送られてくる。フレキシブル基板は送られてきた信号を後段の読み出しバックエンドへ送る。
  \item 電源の供給 \\
  外部からの電源を、センサーとFEチップに供給する。空乏領域を増加させるため、プラナーセンサーには$100\ \si{V}$程度、3Dセンサーには数$\si{V}$のHV(\textbf{H}igh \textbf{V}oltage)をかける。FEチップには、電源供給のために$5.6\ \si{V}$程度のLV(\textbf{L}ow \textbf{V}oltage)を印加する。
  \item モジュールの制御システム(DCS: \textbf{D}etector \textbf{C}ontrol \textbf{S}ystem) \\
  モジュールの温度測定のために2つのNTC(\textbf{N}egative \textbf{T}emperature \textbf{C}oefficient)が配置されている。
\end{itemize}

%------------------------------------------------------------------------------------------------------------------------
\subsection{モジュールキャリア}
\label{sec:carrier}
%------------------------------------------------------------------------------------------------------------------------

モジュールキャリアはモジュールの運搬の際や品質試験を行う際に、モジュールを保護用の容器である (\fref{fig:itkpixelhaiti}のQuadモジュールの銀色部分)。組み立てられたモジュールはFEチップとフレックス基板を繋ぐワイヤー部やセンサーの部分等が剥き出しになっているため、そのままの状態で品質試験を行うのはモジュール破損のリスクを伴う。モジュールキャリアでモジュールを保護することにより、安全に品質試験を行うことができる。

また、モジュールキャリアの別の役割として、モジュール周囲の湿度環境を一定に保つことが挙げられる。運転時に想定される最低温度は$-45\ \si{\degreeCelsius}$のため、品質管理試験ではペルチェ素子を用いた温度制御装置\footnote{KEKにおける次世代ピクセルモジュールの量産では、東工大を中心に開発している温度制御システムを用いる。}を用いて最低$-55\ [\si{\degreeCelsius}]$までモジュールの周囲温度を下げる。その際、ピクセルモジュールに結露が発生すると損傷のリスクを伴う。そのため、キャリア内に乾燥空気を流し込むことで氷点下におけるピクセルモジュールへの結露を防いでいる。

%------------------------------------------------------------------------------------------------------------------------
\section{次世代ピクセルモジュールの組み立て工程}
\label{sec:assemble}
%------------------------------------------------------------------------------------------------------------------------
\begin{figure}[tbp]
  \centering
  \includegraphics[height=2.3cm,keepaspectratio]{module_flow.png}
  \caption[ピクセルモジュールの組み立て工程]{ピクセルモジュールの組み立て工程。 }
  \label{fig:assemble}
\end{figure}


次世代ピクセルモジュールの組み立て工程を\fref{fig:assemble}に示す。組み立て工程はフレキシブル基板とベアモジュールの接着から始まり、ワイヤー配線、パリレン高分子によるコーティング、ワイヤー保護を行いピクセルモジュールが完成する。その後、温度サイクル試験および低温耐久試験において、運転時に想定される温度環境において組み立てたモジュールが運用できるかの試験を行う。本節では、組み立て工程、およびモジュールの温度耐久についての試験についての説明を示す。

\subsubsection*{ベアモジュール・フレキシブル基板の接合}

モジュールの組み立て工程は、組み立て機関に輸送されたベアモジュールとフレキシブル基板の接合から始まる。輸送された各部品の受け取り時の品質試験を行った後、ベアモジュールとフレキシブル基板の接合を行う。専用治具を用いて行うことにより、フレックス基板の位置の交差は$\pm 50\ \si{\micro m}$、平面度は$25\ \si{\micro m}$の精度で接合を行うことができる。

\subsubsection*{ワイヤー配線}

フレキシブル基板とFEチップを電気的に接合し、電源の供給や、FEチップからの信号を読み出すため、フレキシブル基板とFEチップ間をワイヤーで接続する。この組み立て工程をワイヤー配線と呼ぶ。ワイヤーは直径$25\ \si{\micro m}$でのアルミ製であり、1モジュールに対して約$500$本用いられる。ワイヤー配線後からは、モジュールの電気的な読み出しが行うことができるため、これ以降の全ての組み立て工程では読み出し試験を行い正常に動作するかの確認を行う。

\subsubsection*{パリレンコーティング}

モジュールのセンサーとFEチップの端の部分での放電を防ぐこと、湿気や化学物質からの保護を目的としてパリレンコーティングを行う。パリレンはパラキシリレン系ポリマーの略である。パリレンは結晶性が高く絶縁耐力に優れ、周波数に依存せず低い誘電率・誘電正接特性を持っており、湿気や腐食性ガスへの耐性も併せ持つ。


\subsubsection*{ワイヤー保護}

ワイヤーは直径$25\ \si{\micro m}$と非常に細いため、力が加わると損傷してしまう可能性がある。ITkを実装する際、モジュールとケーブルの距離は$2\ \si{mm}$程度のため、モジュールのワイヤーがケーブルに触れてしまい読み出しが正常にできなくなる恐れがある。このような問題を避けるため、\fref{fig:protection}に示すような構造体を用いて、ワイヤーを保護する。

\begin{figure}[tbp]
  \centering
  \includegraphics[height=3.5cm,keepaspectratio]{protection.png}
  \caption[ワイヤー保護用の構造体]{ワイヤー保護用の構造体。左図はワイヤー保護用の構造体を付けたモジュール全体の模式図、右図はモジュール側面から見た断面図を表す。炭素系の素材であるCFRPを用いて構造体を作成する。図中において、緑色はFEチップ、黄色はバンプ、桃色はセンサー、紫色はフレキシブル基板、橙色はワイヤー保護用の構造体を表す。 }
  \label{fig:protection}
\end{figure}

\subsubsection*{温度サイクル}

組み立てたモジュールに対して、ITk実装後にされる特異的な温度変化を想定した温度サイクルを行い、その後もモジュールが正常な応答をするか試験をする。温度変化の際、モジュールの部品間の熱膨張の違いにより熱応力が生じ、それが原因でバンプ接合部に剥がれが生じてしまうことがある。このような温度サイクルによるモジュールの損傷がないことを確認する必要がある。

ITkの運転の切り替えが年間$10$回以上あるため\CID{634}$10$年間の運転を想定すると$100$回以上の熱サイクルにさらされる。
量産における温度サイクル試験では、動作温度範囲$-45\ \si{\degreeCelsius}<T<40\ \si{\degreeCelsius}$の温度サイクルを$10$回、$-55\ \si{\degreeCelsius}<T<60\ \si{\degreeCelsius}$の温度サイクルを$1$回を行う。この$10+1$回のサイクルでは顕著な損傷は出ないと考えられるが、バンプ接合が非常に不良なものを取り除くことができる。
これらの温度サイクルの後にモジュールが正常に動作するかを確認するため、FEチップ回路読み出し試験等を行う。温度サイクルにおけるモジュールの周囲温度の変更は、恒温槽を用いて制御する予定である。

\subsubsection*{低温耐久試験}

ITk運転におけるピクセルモジュールの周囲温度は$-15\ \si{\degreeCelsius}<T<0\ \si{\degreeCelsius}$である。組み立てたモジュールが低温環境下において長時間正常に動作することを確認する試験が低温耐久試験である。
低温耐久試験では、\fref{fig:yomidashi}に示すような温度制御筐体を用いてモジュールの周囲温度を$-15\ \si{\degreeCelsius}$に保ちつつFEチップの回路読み出し試験を行う。読み出し試験は1時間に1度行われる。
\begin{figure}[tbp]
  \centering
  \includegraphics[height=7cm,keepaspectratio]{yomidashi.png}
  \caption[読み出し試験のセットアップ]{読み出し試験のセットアップ。この図は、温度制御筐体の蓋を開けた状態であり、測定時には蓋を閉めてモジュールの周囲を密閉し測定を行う。銀色のフレームがモジュールキャリアであり、その中にモジュールが設置されている。}
  \label{fig:yomidashi}
\end{figure}
%\begin{figure}[tbp]
%  \centering
%  \includegraphics[height=7cm,keepaspectratio]{electrical_system.png}
%  \caption[読み出し試験の全体図]{読み出し試験の全体図。 }
%  \label{fig:electrical-system}
%\end{figure}

長時間放置しつつ読み出し試験を行うため、インターロックシステム、機器の遠隔制御、温度制御筐体の遠隔監視等の技術が必要となる。
%To do: ITk稼働の際にモジュール周囲温度が低い理由を二章に書く。\\
%\url{https://arxiv.org/pdf/2003.00055.pdf}, \\
%\url{https://iopscience.iop.org/article/10.1088/1748-0221/8/10/P10003/pdf}

%------------------------------------------------------------------------------------------------------------------------
\section{品質試験}
\label{sec:QCtest}
%------------------------------------------------------------------------------------------------------------------------

モジュールの各組み立て工程の後に、モジュールが正常に動作するかを確認するために品質試験を行う。\fref{fig:assemble}に示したように、モジュールの外観検査は全ての工程で行われ、FEチップの回路読み出し試験はワイヤー配線後の全ての工程で行われる。本節では、各品質試験項目の詳細を以下に示す。



%------------------------------------------------------------------------------------------------------------------------
\subsection{外観検査}
\label{sec:visualinsp}
%------------------------------------------------------------------------------------------------------------------------
モジュールの表面(フレックス基板側)をカメラを用いて撮影し、モジュールに損傷や汚れ等がないことを目視で確認する。特に、FEチップとフレックス基板を電気的に接合するためのワイヤーの接着位置が正しいか、断線がないかを確認することが重要である。目視で確認する際は、モジュール全体の高解像度画像を36(縦横$6\times6$)分割して得られる拡大画像を用いて細かく検査を行う。また、ワイヤー部分については約$500$本のワイヤーを目視で漏れなく検査することは困難且つ労力を伴うため、ワイヤーの断線や接続部分のずれを自動で検知するアルゴリズムの開発が進んでいる。



%------------------------------------------------------------------------------------------------------------------------
\subsection{平坦性測定}
\label{sec:metrology}
%------------------------------------------------------------------------------------------------------------------------
モジュール上の3次元位置座標を取得することにより、歪み具合や接着剤の厚み等を計算することができる。これにより、接着時のずれや接着剤の厚み等を測定することができる。
%\fref{fig:metrology}に平坦性測定結果を示す。

%\begin{figure}[tbp]
%  \centering
%  \includegraphics[height=7cm,keepaspectratio]{metrology.png}
%  \caption[平坦性測定の結果]{平坦性測定の結果 }
%  \label{fig:metrology}
%\end{figure}


%------------------------------------------------------------------------------------------------------------------------
\subsection{質量測定}
\label{sec:mass}
%------------------------------------------------------------------------------------------------------------------------
質量測定では、モジュール全体の質量を測定する。各工程における質量の差を計算することにより、接着剤の質量やワイヤーの合計質量等を取得することができる。

%------------------------------------------------------------------------------------------------------------------------
\subsection{ワイヤー強度検査}
\label{sec:mass}
%------------------------------------------------------------------------------------------------------------------------

ワイヤー配線により接続したワイヤーの強度を調べるため、専用の機械を用いてワイヤー部分に負荷を与える。


%------------------------------------------------------------------------------------------------------------------------
\subsection{センサー IV特性}
\label{sec:sensoriv}
%------------------------------------------------------------------------------------------------------------------------
センサーの電流-電圧特性を調べることにより、モジュール製造工程におけるセンサーの損傷やHVのショートを確認することができる。
プラナーセンサーでは、漏れ電流が$80\ \si{V}$で$2\ \si{\micro A}$、降伏電圧が$120\ \si{V}$、3Dセンサーについては漏れ電流が$25\ \si{V}$で$2\ \si{\micro A}$、降伏電圧が$35\ \si{V}$程度となるのが想定されうる結果である。また、測定はFEチップからの消費電力による発熱を避けるため、FEチップへのLVを切った状態で行われる。
%Todo: 2章を書きつつ、センサーの詳細を調べ直す。


%------------------------------------------------------------------------------------------------------------------------
\subsection{SLDO VI特性}
\label{sec:sldovi}
%------------------------------------------------------------------------------------------------------------------------
ITk実装時には、モジュールを直列に並べて電源の供給を行う。そのため、各モジュールに対する電源は定電圧ではなく、供給電圧はつなげるモジュールの数に依存してしまう。
FEチップ回路内部で一定の電圧を供給するために、SLDO(\textbf{S}hunt \textbf{L}ow \textbf{D}rop \textbf{O}ut)という制御回路を用いる。SLDO制御回路が供給電流の一部を用いてデジタル回路、アナログ回路の動作電圧を生成し、余剰電流はグランドに捨てられる。FEチップには二つのSLDO制御回路が搭載されている。1つはデジタル回路用、もう1つはアナログ回路用に用いる。\fref{fig:sldoref}にFEチップへの電流と、入力電圧$V_\mathrm{in}$、アナログ回路の出力電圧$V_\mathrm{analog}$およびデジタル回路の出力電圧$V_\mathrm{digital}$の例を示す。
\begin{figure}[tbp]
  \centering
  \includegraphics[height=7cm,keepaspectratio]{sldoref.jpg}
  \caption[FEチップへの電流と、入力電圧$V_\mathrm{in}$、アナログ回路の出力電圧$V_\mathrm{analog}$およびデジタル回路の出力電圧$V_\mathrm{digital}$の関係]{FEチップへの電流と、入力電圧$V_\mathrm{in}$、アナログ回路の出力電圧$V_\mathrm{analog}$およびデジタル回路の出力電圧$V_\mathrm{digital}$の関係\cite{sldo}。}
  \label{fig:sldoref}
\end{figure}
この結果から、SLDO回路によりアナログ回路およびデジタル回路からの出力電圧は一定であり、制御回路が正常に動作していることがわかる。

さらに、SLDO VI特性についての品質試験を行う際には、FEチップが低温においても正常に動作するかの試験も行う。モジュールの周囲温度を$-35\ \si{\degreeCelsius}$にし、デジタル回路の読み出し試験を行いモジュールが低温環境で動作することを確認する。正常に動作しない場合は$15\ \si{\degreeCelsius}$ずつ温度を上げて再び試験を行い、正常に動作を始める温度を記録する。温度を記録する際には、\tref{tab:gradesldo}に示すように各温度に対する階級を表す数値を用いて値の入力を行う。

\begin{table}[tbp]
  \begin{center}
    \caption[モジュール起動温度に対する階級値]{モジュール起動温度に対する階級値。}
    \label{tab:gradesldo}
    \begin{tabular}{|c|c|}
    \hline
      温度[$\si{\degreeCelsius}$] & 階級値 \\
    \bhline{1.5pt}
     $-35$ & $1$ \\
    \hline
     $-35 < T \leq -20$ & $2$ \\
    \hline
     $-20 < T \leq -5$ & $3$ \\
    \hline
     $-5 < T \leq 10$ & $4$ \\
    \hline
     $10 < T$ & $5$ \\
    \hline
    \end{tabular}
  \end{center}
\end{table}


%------------------------------------------------------------------------------------------------------------------------
\subsection{読み出し試験}
\label{sec:electricaltest}
%------------------------------------------------------------------------------------------------------------------------
モジュールに通電し、正常に読み出しができるか確認する。読み出し試験はITk運転時の温度環境を想定し、モジュールの周囲温度を変えつつ試験を行う。そのため、読み出し試験の際には低温耐久試験と同様に\fref{fig:yomidashi}の温度制御筐体を用いて試験を行う。設定温度は、低温におけるモジュール起動試験では$-35\ \si{\degreeCelsius}$ (正常に起動できない場合は$15\ \si{\degreeCelsius}$ずつ温度を上げて試験)、Threshold測定やToT測定等の通常の読み出し試験では$-20\ \si{\degreeCelsius}, 20\ \si{\degreeCelsius}$である。

読み出し試験に用いるDAQ(\textbf{D}ata \textbf{A}c\textbf{q}uisition)として、YARR(\textbf{Y}et \textbf{A}nother \textbf{R}apid \textbf{R}eadout)を使用する。YARRとはピクセルモジュール用に開発された、PCI Express(PCIe) 接続を用いた読み出しシステムである。YARRを用いて行う、モジュールの読み出し試験の項目を以下に示す。

\begin{itemize}
  \item Digitalスキャン \\
  各ピクセルについてのデジタル回路の応答を確認する。デジタル回路に試験用パルスを入射し、信号の応答数を測定する。
  \item Analogスキャン \\
  各ピクセルについてのアナログ回路の応答を確認する。アナログ回路に試験電荷を入射し、信号の応答数を測定する。
  \item Tresholdスキャン \\
  各ピクセルのThresholdを確認する。試験電荷を用いてSカーブのフィッティング(\ref{sec:tuning}節におけるThresholdスキャンと同様の手法)を行い、Threshold値やノイズを測定する。
  \item ToTスキャン \\
  一定の試験電荷を各ピクセルに100回入射させ、その試験電荷に対するToTの値の測定を行う。
  \item Noiseスキャン \\
  各ピクセルのノイズを確認する。試験電荷を用いず、クロックによるトリガーで$t\ [\si{sec}]$の測定を行い、応答率を求める。Noiseスキャンの応答率$\mathrm{NoiseOccupancy}$は、取得信号数$n_0$を用いて\eref{eq:noise}で定義される。
  \begin{equation}
  \label{eq:noise}
  \mathrm{NoiseOccupancy} = \frac{n_0}{t\ [\si{sec}]}\times 2.5\ [\si{nsec}]
  \end{equation}
  \item Sourceスキャン \\
  放射線を照射し、各ピクセルの応答を確認する。これにより、シリコンセンサーとFEチップ間の接続の不具合等も確認することができる。
\end{itemize}

また、これらのスキャン項目に加え、ピクセルのチューニングも行うことがある。チューニングすることにより、各ピクセルのThreshold値、および基準電荷量に対するToTの値を目標値に近づけることができる。チューニングに関するスキャン項目を以下に示す。
\begin{itemize}
  \item Global threshold チューニング \\
  FEチップ全体のピクセルにおけるThresholdを一括チューニングする。
  \item Pixel threshold チューニング\\
  各ピクセル毎に Thresholdの値をチューニングする。
  \item Global ToT チューニング \\
  FEチップ全体のピクセルにおけるToT値を一括チューニングする。
  \item Pixel ToT チューニング \\
  各ピクセル毎にToT値をチューニングする。
\end{itemize}

読み出し試験の測定後、各ピクセルが正常に応答しているかを判断する必要がある。そのために決定された評価基準を\tref{tab:pixel-failure}に示す。試験結果の解析の際には、あるピクセルモジュールにおける不良ピクセルの分布および合計不良ピクセル数が記録される。

%\begin{table}[tbp]
%  \begin{center}
%    \caption[Powering overview for the on-detector system]{Powering overview for the on-detector system \cite{itk}}
%    \label{tab:powering}
%    \begin{tabular}{|l|l||l|}
%    \hline
%      \multirow{5}{*}{on-module} & モジュールへの電流値 & $5.6\ \si{A}$ \\
%    \cline{2-3}
%      & モジュールへのLV値 & $1.4\ \si{V}$ \\
%    \cline{2-3}
%      & モジュールの消費電力 & $7.85\ \si{W}$ \\
%    \cline{2-3}
%     & センサー & $< 0.1\ \si{W/cm^2}$ \\
%    \hline
%     \multicolumn{2}{|l||}{DCS power} & $0.15$-$0.28\ \si{W}$ per quad module \\
%    \hline
%     \multicolumn{2}{|l||}{Cooling system capability} & $0.7\ \si{W/cm^2}$ \\
%    \hline
%    \end{tabular}
%  \end{center}
%\end{table}


\begin{table}[tbp]
  \begin{center}
    \caption[ピクセル解析の評価基準一覧]{ピクセル解析の評価基準一覧 \cite{lingxin}。表中のOccupancyは入射信号数を$n_i$、取得信号数を$n_0$としたとき、$n_i/n_0\times 100\%$で表される値である。これらの基準一覧は不良ピクセルの評価であり、全ての基準に当てはまらないピクセルが良好と判断される。}
    \label{tab:pixel-failure}
    \begin{tabular}{|l|l||l|}
    \hline
      評価名 & 読み出し試験項目 & 評価基準 \\
    \bhline{1.5pt}
      Digital Dead & Digital Scan & $\mathrm{Occupancy}<1\si{\%}$ of injections \\
    \hline
      Digital Bad & Digital Scan & $\mathrm{Occupancy}<98\si{\%}$ or $\mathrm{Occupancy}>102\si{\%}$ of injections \\
    \hline
      \multirow{2}{*}{Merged Bump} & Analog Scan & $\mathrm{Occupancy}<1\si{\%}$ of injections \\
       & Crosstalk Scan & $\mathrm{Occupancy}<80\si{\%}$ of $25\ \si{ke}$ injections \\
    \hline
      Analog Dead & Analog Scan & $\mathrm{Occupancy}<1\si{\%}$ of injections \\
    \hline
      Analog Bad & Analog Scan & $\mathrm{Occupancy}<98\si{\%}$ or $\mathrm{Occupancy}>102\si{\%}$ of injections \\
    \hline
      \multirow{2}{*}{Tuning Failed} & Threshold Scan & Sカーブのフィットが失敗 \\
       & ToT Test & ToT$=0$ or $14$ \\
    \hline
      Noisy & Noise Scan & $\mathrm{NoiseOccupancy}<10^{-6}$ \\
    \hline
      Disconnected Bump & Source Scan & $\mathrm{Occupancy}<1\si{\%}$ of mean Occupancy \\
    \hline
      High Crosstalk & Crosstalk Scan & $\mathrm{Occupancy}>0$ with $25\ \si{ke}$ injection \\
    \hline
    \end{tabular}
  \end{center}
\end{table}

%------------------------------------------------------------------------------------------------------------------------
\subsection{モジュール特性}
\label{sec:module-prop}
%------------------------------------------------------------------------------------------------------------------------
モジュールの特性として、以下のようのものが定義されている。
\begin{itemize}
  \item FEチップの種類 \\
  モジュールの部品となっているFEチップの種類を入力する。現在行っている試作器はRD53Aであり、ITkに向けたモジュール量産ではITkpix\_v2を用いる予定である。
  \item Thickness \\
  センサーの厚さの情報を入力する。この特性に入力する値としては"thin"と"thick"があり、それぞれセンサーの厚みが$150\ \si{\micro m}$と$300\ \si{\micro m}$である。
  \item Roof \\
  モジュールのワイヤー部を保護する構造体の有無を記録する。
  \item IrefTrim値\\
  全てのDAC\footnote{D/Aコンバーターとも呼ぶ。}(\textbf{D}igital \textbf{A}nalog \textbf{C}onverter)は"IREF"と呼ばれるグローバルなレファレンスから$4\ \si{\micro A}$の電流を生成する。Irefの値はワイヤー配線の際に4bitの値で決定され、\fref{fig:irefbit}のようにIref値を変更することによりSLDO制御回路により出力される電流値が変化する。ワイヤー配線とIref値の関係を\fref{fig:iref-detail}に示す。
  \item プルアップ抵抗値\\
  プルアップ抵抗とは、電子回路の入力端子に接続される抵抗であり、スイッチがオフの状態の時に電圧が一定の値を安定して維持できるようにするためのものである。アナログ回路への電源VDDAとプルアップ抵抗値についての関係を\tref{tab:pull-up}に示す。
%  Compare VREF\_A\_Trim at DAC count 16 with VDDA on hybrid after power-up \\
%  For the chip to start up, VDDA > 1.14A\\
%  If this is lower, you need to add a pull-up resistor!\\
%  NTCに関係がある? \\
  \item ベアモジュールとフレキシブル基板の向き \\
  ベアモジュールとフレキシブル基板の向きが正しいことを保証する。
  %ベアモジュールの見た目は、$180\si{\degree}$回転対称になっている。そのため、フレックス基板とベアモジュールの貼り付け工程で誤って$180\si{\degree}$回転した状態で接合してしまうことがある。この誤りは、ワイヤー配線後に行う読み出し試験の際に初めて確認される。その場合は、ベアモジュールとフレックス基板の向きについての特性を書き直す必要がある。
\end{itemize}

\begin{figure}[tbp]
  \centering
  \includegraphics[height=7cm,keepaspectratio]{irefbit.png}
  \caption[Iref値と電流の関係]{Iref値と電流の関係 \cite{irefbit}。Iref値は$4\ \si{bit}$で与えられ、その値により電流値が決まる。Iref値により電流値を$4\ \si{\micro A}$に制御することが目標のため、この図の場合はIref値を9にするのが良い。}
  \label{fig:irefbit}
\end{figure}
\begin{figure}[tbp]
  \begin{minipage}[b]{0.45\linewidth}
    \centering
    \includegraphics[keepaspectratio, scale=0.25]{iref_module.png}
  \end{minipage}
  \begin{minipage}[b]{0.45\linewidth}
    \centering
    \includegraphics[keepaspectratio, scale=0.5]{iref_detail.png}
  \end{minipage}
  \caption{クアッドモジュールのIrefTrim部分を表す部分(左図)とワイヤーの配線とIref値の関係(右図)\cite{lingxin}。ピンク色のワイヤーの配置により4bitのIref値を表すことができる。}
  \label{fig:iref-detail}
\end{figure}


%\begin{figure}[tbp]
%  \centering
%  \includegraphics[height=6cm,keepaspectratio]{pullup.png}
%  \caption[pull-up Resistor]{Pull-up Registor}
%  \label{fig:pillup}
%\end{figure}

\begin{table}[tbp]
  \begin{center}
    \caption[プルアップ抵抗値と電源の関係]{プルアップ抵抗値と電源の関係。}
    \label{tab:pull-up}
    \begin{tabular}{|c|c|c|c|}
    \hline
      VDDA [\si{V}] & Pull-up Resistorが必要か & Pull-up Resistorの値 [$\si{\ohm}$] & VDDAの増加量\\
    \bhline{1.5pt}
     $\mathrm{VDDA}\leq 1.09$ & 必要 & 150 & 0.1 \\
    \hline
     $1.09 < \mathrm{VDDA} \leq 1.14$ & 必要 & 300 & 0.05 \\
    \hline
     $1.14 > \mathrm{VDDA}$ & 不要 & n/a & n/a \\
    \hline
    \end{tabular}
  \end{center}
\end{table}

上記に示したモジュールの特性のうち、FEチップの種類等の情報はモジュールの組み立てを始めると同時に決まるものである。しかし、Iref値やプルアップ抵抗値についての特性はワイヤー配線後に初めて確認できるものであるため、品質試験を行う際に並行して確認する必要がある。

また、ベアモジュールの見た目は、$180\si{\degree}$回転対称になっている。そのため、フレックス基板とベアモジュールの貼り付け工程で誤って$180\si{\degree}$回転した状態で接合してしまうことがある。この誤りは、ワイヤー配線後に行う読み出し試験の際に初めて確認される。そのため、ベアモジュールとフレックス基板の向きが正しいことを再確認する必要がある。

%------------------------------------------------------------------------------------------------------------------------
\section{量産における試験結果管理}
\label{sec:production-manage}
%------------------------------------------------------------------------------------------------------------------------
%各モジュールに対して、各組み立て工程および温度耐性についての品質試験を行い合計$30$個程度の試験を行う。
%試験項目によって結果データの形式は異なる。各品質試験のデータの形式およびデータサイズを\tref{tab:DBdata}に示す。
%\begin{table}[tbp]
%  \begin{center}
%    \caption[品質試験のデータ]{品質試験のデータ}
%    \label{tab:DBdata}
%    \begin{tabular}{|l||l|c|c|}
%    \hline
%      試験項目 & 内容 & データ形式 & データサイズ\\
%    \bhline{1.5pt}
%     読み出し試験 & FEチップ上の各ピクセルの結果 & JSON file & $629\ \si{MB}$ \\
%    \hline
%    \end{tabular}
%  \end{center}
%\end{table}
%品質試験結果に加えて、ワイヤー配線をした後に確認できる、モジュール特性についてのデータも適切に管理する必要がある。

ITkのために世界各地の組み立て機関においてモジュールの量産およびそのための品質管理試験を行う。各組み立て機関では$\mathcal{O}(100)\sim \mathcal{O}(1000)$のモジュールの量産を行う予定であり、最も量産数が多いのは日本の高エネルギー加速器研究機構(KEK)における約$2000$個のモジュールの量産である。各モジュールに対して、各組み立て工程および温度耐性についての合計$30$個程度の品質試験を行い結果を管理する。さらに、ワイヤー配線後に確認できるモジュール特性についてのデータも適切に管理する必要がある。
%KEKにおける量産についてデータサイズは$??\ \si{TB}$になると想定される

ITkの製造に関係するモジュール情報や品質試験の結果は、チェコに設置されている中央のデータベースに保存する必要がある。そのため、各研究機関においてモジュール情報と品質試験結果を統一的に管理する必要がある。しかし、各モジュールにおいて合計$30$個程度の試験項目がありデータの形式が異なること、さらに各研究機関で組み立てるモジュール数が多いことから、各研究機関独自のシステムを用いてデータ管理すると以下のような問題が想定される。
\begin{itemize}
  \item データの不整合 \\
  各研究機関において独自のシステムを用いると、試験結果の管理方法が異なることから他の機関との結果の比較が困難になる。モジュールを他研究機関に送った際に、輸送中にモジュールに損傷がないことを確認するために受け取り試験(レセプション試験)を行う。輸送前後の試験結果を比較を行うが、その際に試験結果の形式が異なると比較前に結果を整形する必要があり、試験機関の数だけ整形しなければならないため非常に面倒である。試験結果の各パラメータや試験項目が全ての研究機関において共通となるために、統合的なシステムが必要となる。
  \item データの重複 \\
  試験結果やモジュール情報を共有した際、既にそのデータが存在すると使用しているPCのディレクトリ構造の違いにより別のデータと認識してしまいデータの重複が発生してしまう可能性がある。そのため、別の研究機関や異なるサーバーでデータを共有する際に、適切に管理する必要がある。
\end{itemize}

上記の問題を解決するために、各モジュール組み立て機関において適切にデータ管理を行うことを目的として、データベースシステムの開発を行っている。
本研究では、これまで読み出し試験用に開発されていたデータベースシステムを改良し、次世代モジュール量産におけるデータ管理を全てをサポートできるよう諸機能の開発を行った。
次章からデータベースシステムを説明する。

\newpage
