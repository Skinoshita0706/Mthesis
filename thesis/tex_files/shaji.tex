\chapter*{謝辞}

本研究を進めるにあたり、多くの方々にご助力いただけましたことを心より感謝申し上げます。

指導教員である陣内修先生には、研究


電荷較正については、特にKEKの方々と議論をしながら開発を行いました。津野総司さんには、現在の電荷較正

データベースシステムについては、大阪大学および都立大学の方々と議論しながら開発を行いました。廣瀬穰さんには開発方針をはじめとする、研究を進めるに当たって重要な助言、議論をしてもらいました。品質試験結果登録用GUIの開発者の大西裕二さん、藤田幸子さん、大島恵里香さんとは特にコミュニケーションをとり、開発内容についての議論を行いました。また、KEKで試作器の開発を行っている外川学さんには諸機能についてのフィードバックをいただき、実際にKEKに行った時にお世話になりました。ありがとうございました。

また、本テーマにおいては日本人以外の研究者の方々と議論する機会も多くいただきました。Monika Wielersさんをはじめとするピクセルグループのデータベースの担当者の方々と、データベースの構造について多く議論しました。さらに、中央データベースと通信する諸機能については、中央データベースAPIのpython packageの開発者であるGiordon Starkには多くのアドバイスをいただきました。また、海外の数多くの研究者の方にグループチャットにて様々なフィードバックをいただきました。世界的に多くの研究者の方々と議論をできたことは、研究面のみではなく言語面でも自分にとって大きな財産になりました。ありがとうございました。

さらに、陣内研究室の皆様に感謝いたします。研究室内での発表

最後に、私の研究生活を支えてくださった皆様と家族に心から感謝いたします。本当にありがとうございました。
