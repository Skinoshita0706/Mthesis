\documentclass[a4paper,11pt,oneside,openany]{jsbook}
%
\usepackage{amsmath,amssymb}
\usepackage{bm}
\usepackage[dvipdfmx]{graphicx}%,draft
\usepackage{graphicx}
\usepackage{subfigure}
\usepackage{verbatim}
\usepackage{wrapfig}
\usepackage{ascmac}
\usepackage{makeidx}
\usepackage[dvipdfmx]{hyperref,graphicx}
\usepackage{pxjahyper}
%\hypersetup{
%	colorlinks=false, % リンクに色をつけない設定
%	bookmarks=true, % 以下ブックマークに関する設定
%	bookmarksnumbered=true,
%	pdfborder={0 0 0},
%}
\usepackage{siunitx}
\usepackage{lineno}
\pagewiselinenumbers 
%
\makeindex
%
\setlength{\textwidth}{\fullwidth}
\setlength{\textheight}{40\baselineskip}
\addtolength{\textheight}{\topskip}
\setlength{\voffset}{-0.55in}
%
\newcommand{\diff}{\mathrm{d}}  %微分記号
\newcommand{\divergence}{\mathrm{div}\,}  %ダイバージェンス
\newcommand{\grad}{\mathrm{grad}\,}  %グラディエント
\newcommand{\rot}{\mathrm{rot}\,}  %ローテーション
%
\def\maru#1{{\rm\ooalign{\hfil\lower.168ex\hbox{#1}\hfil \crcr\mathhexbox20D}}}
\makeatletter
\newcommand{\figcaption}[1]{\def\@captype{figure}\caption{#1}}
\newcommand{\tblcaption}[1]{\def\@captype{table}\caption{#1}}
%
\newcommand{\eref}[1]{式~(\ref{#1})}
\newcommand{\fref}[1]{図~\ref{#1}}
\newcommand{\tref}[1]{表~\ref{#1}}
%
\title{ATLASピクセル検出機の電荷補正方法の最適化と\\新型ピクセル検出機量産の品質試験結果管理システムの開発}
\author{東京工業大学 理学院物理学系物理学コース 陣内研究室\\木下怜士(20M00395)}
\date{2021年7月30日}
\begin{document}
%
%
\maketitle
%
%
\frontmatter
%
\addcontentsline{toc}{chapter}{概要}
\chapter*{Abstract}

The Large Hadron Collider (LHC) is the world's highest energy particle collider, which designed to provide proton-proton collisions with center of mass energy of $14\ \si{TeV}$. The ATLAS is a general purpose detecctor at the LHC. The purposes of the ATLAS are to investigate precise measurements of the Standard Model and searches for the physics beyond the Standard Model.

The pixel detector is installed in the innermost layer of ATLAS.
%The calibration of the pixel detector fulfills tuning the ASIC configuration parameters for establishing the best operational settings.
The calibration of the pixel detector fulfills tuning ASIC configuration parameters.
Parameters can be changed because of the radiation damege so the calibration need to be done frequently.
The calibration results can contain two main problems: ASIC injects incorrect charges and data is lost.
%We need to correct these probrems before storing results to database.
This thesis reports the development of an analysis tool that automatically removes the problems by using appropriate exception handling.
The tool is used for the latest calibration data obrained in September 2021. It found that the calibration results can not predict correct small charges.
Evaluating the effect on the measurement data including small charges, and it is expected to be small for MIP measurement, which is the main target of the pixel detector.
%As a result, it was decided not to correct the charge amount obtained from the charge calibration equation because the effect on clusters containing small charge amounts was expected to be small for MIP, which are the main measurement target of the pixel detector.

The LHC plans to be upgraded so that it will increase the instantaneous luminosity. The upgrade is called HL-LHC(High-Luminosity LHC).
The HL-LHC will have an instantaneous luminosity 5-7 times larger than the current LHC.
This is expected to improve the measurement accuracy of the Standard Model and the sensitivity of the search for new physics.
With the increase of instantaneous luminosity, higher performance of the detector is required due to the increase of data rate and hit occupancy.
In addition, the current ATLAS pixel detectors are expected to reach their design radiation tolerance.
So the inner detectors will be replaced to new detector called ITk(Inner Tracker).

For the production of ITk, about 10,000 pixel modules will be mass-produced, and quality tests will be conducted for all modules.
In order to ensure efficient mass production and integrated pixel module selection, a system for integrated management of quality control tests' results is required.
For this purpose, an integrated management system called \textit{local database} is being developed mainly by Tokyo Institute of Technology.
This thesis also reports a management features such as displaying test results, registering a new module, and synchronizing data with central database.

Integrating these features with those of the previous study, all the basic features required for the mass production are now in place.
However, it was confirmed that some of the functions require very long processing times.
Such a long processing time is very detrimental to the user's satisfaction and needs to be improved.
%Therefore, we pointed out ways to improve the process by subdividing the process content, which is an issue for the future.

\newpage

\chapter*{概要}

フランスとスイスの国境にある欧州原子力研究機構(CERN)に建設された陽子衝突加速器LHCでは、世界最高エネルギーである重心系エネルギー$14\ \si{TeV}$で陽子同士を衝突させることができる。LHCの衝突点の1つに設置されているATLASでは、陽子衝突からの崩壊生成粒子を測定し、標準模型の精密測定やそれを超える新物理の探索を行っている。

ATLASの最内層に設置されているシリコンピクセル検出器は、放射線損傷による影響を補正するために頻繁に電荷較正を行う必要がある。電荷較正結果には、ASICが正しい電荷を生成できないことや、データの欠損等の問題が含まれることがあるため、これら値を適切な値に補完する必要がある。本研究では、電荷較正結果により適切な例外処理を行うよう、例外をアルゴリズムとして抽出・処理する解析ツールを開発した。
開発した解析ツール用いて、2021年9月に取得した最新の電荷較正データに対して電荷較正およびその補正を行った。最新の電荷較正データを用いて作成した結果から、小さい電荷量について電荷較正が理想的に行えていないことがわかった。最新の電荷較正データを用いて、小さい電荷量が測定データに与える影響の評価を行った。その結果、ATLASピクセル検出器における主な測定対象であるMIP粒子に対して、小さい電荷量を含むクラスターへの影響は小さいと予想され、電荷較正式から得られる電荷量の補正は行わないことに決定した。

また、LHC高輝度化(HL-LHC)に向けたATLAS検出器アップグレードのため、次世代ピクセル検出器の開発および量産の準備を進めている。HL-LHCでは、瞬間ルミノシティが現在のLHCの$5$-$7$倍、積分ルミノシティが$2025$年までの約$10$倍になり、標準模型の測定精度の向上や新物理の探索感度の向上が期待される。瞬間ルミノシティの増加に伴うデータレートの増加やヒット占有率の増加により、検出器に対してより高い性能が要求される。さらに、現行ATLASピクセル検出器は2025年における稼動後、設計放射線耐性に到達するため、再内層に設置されている内部飛跡検出器の総入れ替えを予定している。HL-LHCのために新しく設置する内部飛跡検出器をITk (Inner Tracker)と呼ぶ。

ITkの製造のために、次世代ピクセルモジュール約10000台を量産し、全てのモジュールに対して品質試験を行う予定となっている。効率の良い量産と統合されたピクセルモジュール選定を行うために、品質試験結果を統合管理するシステムが必要となる。そのために、東工大を中心としてローカルデータベースという統合管理システムの開発が進められている。
本研究では、先行研究において開発された読み出し試験についての管理機能に加え、外観鑑別や形状測定などの試験項目についての管理機能、モジュール登録機能、試験結果の共有機能の開発を行った。

以上の機能を新たに実装することにより、ピクセルモジュールの次世代器量産における品質試験結果管理に必要な機能の基本的な部分が全て揃った。しかし、一部の機能において非常に長い処理時間を必要とすることが確認された。このような長い処理時間は使用者の満足度が非常に悪くなるため、改善が必要である。そのため、処理内容を細分化することにより改善方法を指摘し、今後の課題とした。



%世界最高エネルギーでの陽子衝突加速器LHCで新物理の発見を目指すATLAS実験のピクセル検出器においては、電荷較正の結果にデータの欠損や較正の失敗が含まれると、実測およびシミュレーションに影響を及ぼすため、較正結果を評価し再較正を行う必要がある。本研究では、再較正の際により適切な欠損の補完処理を行うよう、例外をアルゴリズムとして抽出・処理する自動解析ツールを開発した。
%
%また、LHC高輝度化に向けたATLAS検出器アップグレードのため、新型ピクセル検出器の開発および量産の準備を進めている。検出器の品質管理のために、組立工程において様々な試験を行う。本研究では、効率の良い量産と統合されたモジュール選定のために、先行する読み出し試験についての管理機能に加え、外観鑑別や形状測定などの試験項目についての管理機能、モジュール登録機能、試験結果の共有機能の開発を行った。

\tableofcontents
%
\mainmatter
%
\chapter{序論}


\section{素粒子標準模型}

\section{LHC}


\section{ATLAS実験}



\section{HL-LHCアップグレード}



\newpage

%----------------------------------------------------------------------------
\chapter{シリコンピクセル検出器}
\label{sec:chap2}
%----------------------------------------------------------------------------
本研究で使用するピクセル検出器は、シリコンを用いた半導体検出器である。本章では、半導体検出器の一般論と、ATLAS実験で用いられているピクセル検出器、およびHL-LHCで用いられる新型ピクセル検出器について説明する。





%----------------------------------------------------------------------------
\section{半導体検出器の一般論}
\label{sec:handoutai}
%----------------------------------------------------------------------------
結晶構造を持つ物質は、その電気的な性質から導体、半導体、絶縁体に大別される。







%----------------------------------------------------------------------------
\section{ピクセル検出器}
\label{sec:pixelkenshutuki}
%----------------------------------------------------------------------------











%----------------------------------------------------------------------------
\section{現行ピクセル検出器}
\label{sec:genkoupixel}
%----------------------------------------------------------------------------






%----------------------------------------------------------------------------
\section{新型ピクセル検出器}
\label{sec:singatapixel}
%----------------------------------------------------------------------------





\newpage

%------------------------------------------------------------------------------------------------------------------------
\chapter{現行ピクセルモジュールの電荷較正}
\label{sec:chap3}
%------------------------------------------------------------------------------------------------------------------------

FEチップから得られるToT (Time over Threshold)を荷電粒子がシリコンセンサーに落とす電荷量に較正する必要がある。本章では、電荷較正のための試験電荷生成回路の詳細について説明し、その後に電荷較正手法について述べる。

%------------------------------------------------------------------------------------------------------------------------
\section{アナログ回路}
\label{sec:analog}
%------------------------------------------------------------------------------------------------------------------------
\fref{fig:analog}にFE-I3のアナログ回路の概略図を示す。この図は1つのピクセルに対するアナログ回路であり、FEチップ上の$18\times160$個のピクセルに対して同様の回路が搭載されている。センサーにおいて生成された電子による信号ををFEチップにおいて送信するために、それらはバンプにより接合されている。生成された電子は内部電位によりバンプに向かってドリフトし、その電子により電極表面の内部電位が変化し、バンプに電流が流れる。その電流によるアナログ信号をバンプの接合部からFEチップへ送りプリアンプで整形および増幅を行う。キャパシタ$C_\mathrm{F}$は信号により充電され、電荷量に依らない一定のフィードバック電流によって放電される。\fref{fig:analog}の$8\ \si{bit}$のIF DACによってFEチップ全体のピクセルについてのフィードバック電流の増幅率の調整を行い、$3\ \si{bit}$のFDACを用いてピクセルごとのフィードバック電流の増幅率の調整を行う。これにより、バンプからの信号は\fref{fig:sannkakuha}のように三角波となり、その波高は入射電荷量によって決定される。整形および増幅された信号をToTにデジタル変換し、後段のフレキシブル基板へ信号を送る。

一方で、Thresholdの測定や電荷較正のために用いる電荷はFEチップ内の回路で生成する。FEチップにおいて試験電荷を生成するために、電圧$V_\mathrm{cal}$を自由に設定できる回路と2つのキャパシタ$C_\mathrm{low},\ C_\mathrm{high}$が搭載されている。試験電荷生成のために、$C_\mathrm{low}=8\ \si{fF}$のキャパシタを用いる場合と、$C_\mathrm{low}+C_\mathrm{high}=40\ \si{fF}$の合成キャパシタを用いる場合がある。$C_\mathrm{low}+C_\mathrm{high}$の合成キャパシタを用いる場合は生成した試験電荷から得られるToTが約$10\%$小さく出力されることがわかっている。そのため、Thresholdの測定や電荷較正を行う際には、$C_\mathrm{low}$のキャパシタを用いて試験電荷の生成を行う。

\begin{figure}[tbp]
  \centering
  \includegraphics[height=9cm,keepaspectratio]{analog2.png}
  \caption[FEI3アナログ回路の概略図]{FEI3アナログ回路の概略図。電荷較正やThresholdスキャンのための試験電荷は$V_\mathrm{cal}$とキャパシタ($C_\mathrm{low},\ C_\mathrm{high}$)の組み合わせによって生成される。}
  \label{fig:analog}
\end{figure}

\begin{figure}[tbp]
  \centering
  \includegraphics[height=6cm,keepaspectratio]{calibeq2.png}
  \caption[ToTと荷電粒子がシリコンセンサーに落とす電荷量$Q$の概念図]{ToTと荷電粒子がシリコンセンサーに落とす電荷量$Q$の概念図。図中の青線と赤線はある電荷量$(Q_1>Q_2)$がToTに変換される概念図である。図中上半分の三角波はアナログ回路のフィードバック回路にて整形・増幅された信号であり、その信号がThreshold値を超える時間であるToTに変換される。三角波立ち上がりはピークまでに約$40\ \si{ns}$になるよう調整されるため電荷量により傾きが異なるが、立ち下がりはフィードバック回路によって制御されるため、電荷量に依存せず一定の傾きである。}
  \label{fig:sannkakuha}
\end{figure}



%------------------------------------------------------------------------------------------------------------------------
\section{電荷較正手法}
\label{sec:calibway}
%------------------------------------------------------------------------------------------------------------------------
ピクセルモジュールの出力であるToTを較正し、荷電粒子が落とした電荷量に変換する方法について説明する。各ピクセル間の差異を少なくするために、ToTの較正を行う前に、ThresholdやToTを目標値になるようチューニングを行う必要がある。以下ではチューニングと電荷較正の方法について説明する。


%------------------------------------------------------------------------------------------------------------------------
\subsection{チューニング}
\label{sec:tuning}
%------------------------------------------------------------------------------------------------------------------------
各ピクセルにおけるThresholdと、ある基準電荷量の信号に対するToTを任意の値に調整するためにFEチップのチューニングを行う。Run2におけるThresholdおよびMIP相当の参照電荷量に対応するToTの目標値をそれぞれ\tref{tab:thresholdtuning}、\tref{tab:tottuning}に示す。さらに、2022年3月から始まるRun3では、B-LayerのThresholdの目標値は$3500\ \si{e}$であり、ToTの目標値はMIP相当の参照電荷量である$20\ \si{ke}$に対して$18\ \si{ToT}$、IBLのThresholdの目標値は$1500\ \si{e}$であり、ToTの目標値はMIP相当の参照電荷量である$16\ \si{ke}$に対して$10\ \si{ToT}$である。

\begin{table}[tbp]
  \begin{center}
    \caption[各LayerにおけるThresholdの値]{各LayerにおけるThresholdの目標値。1行目の括弧内の数字はRun2における積分ルミノシティを表す。}
    \label{tab:thresholdtuning}
    \begin{tabular}{|l||r|r|r|r|}
    \hline
      Layer名  & 2015年($4\ \si{fb^{-1}}$) & 2016年($39\ \si{fb^{-1}}$) & 2017年($50\ \si{fb^{-1}}$) & 2018年($63\ \si{fb^{-1}}$) \\
    \bhline{1.5pt}
      IBL & $2500\ \si{e}$ & $2500\ \si{e}$ & $2500\ \si{e}$ & $2000\ \si{e}$ \\
    \hline
      B-Layer(中央) & $3500\ \si{e}$ & $3500\ \si{e}$ & $5000\ \si{e}$ & $4300\ \si{e}$ \\
    \hline
      B-Layer(前方) & $3500\ \si{e}$ & $3500\ \si{e}$ & $5000\ \si{e}$ & $5000\ \si{e}$ \\
    \hline
      Layer1 & $3500\ \si{e}$ & $3500\ \si{e}$ & $3500\ \si{e}$ & $3500\ \si{e}$ \\
    \hline
      Layer2 & $3500\ \si{e}$ & $3500\ \si{e}$ & $3500\ \si{e}$ & $3500\ \si{e}$ \\
    \hline
      Disk & $3500\ \si{e}$ & $3500\ \si{e}$ & $4500\ \si{e}$ & $3500\ \si{e}$ \\
    \hline
    \end{tabular}
  \end{center}
\end{table}


\begin{table}[tbp]
  \begin{center}
    \caption[各LayerにおけるToTのチューニングの値]{各LayerにおけるToTの目標値。1行目の括弧内の数字はRun2における積分ルミノシティを表し、それ以降の括弧内は各ToTに対する電荷量であり、この値はMIP粒子がセンサーに落とす電荷量を表す。}
    \label{tab:tottuning}
    \begin{tabular}{|l||r|r|r|r|}
    \hline
      Layer名  & 2015年($4\ \si{fb^{-1}}$) & 2016年($39\ \si{fb^{-1}}$) & 2017年($50\ \si{fb^{-1}}$) & 2018年($63\ \si{fb^{-1}}$) \\
    \bhline{1.5pt}
      IBL & $10 \mathrm{ToT}\ (16\ \si{ke})$ & $8 \mathrm{ToT}\ (16\ \si{ke})$ & $8 \mathrm{ToT}\ (16\ \si{ke})$ & $10 \mathrm{ToT}\ (16\ \si{ke})$ \\
    \hline
      B-Layer & $30 \mathrm{ToT}\ (20\ \si{ke})$ & $18 \mathrm{ToT}\ (20\ \si{ke})$ & $18 \mathrm{ToT}\ (20\ \si{ke})$ & $18 \mathrm{ToT}\ (20\ \si{ke})$ \\
    \hline
      Layer1 & $30 \mathrm{ToT}\ (20\ \si{ke})$ & $30 \mathrm{ToT}\ (20\ \si{ke})$ & $30 \mathrm{ToT}\ (20\ \si{ke})$ & $30 \mathrm{ToT}\ (20\ \si{ke})$ \\
    \hline
      Layer2 & $30 \mathrm{ToT}\ (20\ \si{ke})$ & $30 \mathrm{ToT}\ (20\ \si{ke})$ & $30 \mathrm{ToT}\ (20\ \si{ke})$ & $30 \mathrm{ToT}\ (20\ \si{ke})$ \\
    \hline
      Disk & $30 \mathrm{ToT}\ (20\ \si{ke})$ & $30 \mathrm{ToT}\ (20\ \si{ke})$ & $30 \mathrm{ToT}\ (20\ \si{ke})$ & $30 \mathrm{ToT}\ (20\ \si{ke})$ \\
    \hline
    \end{tabular}
  \end{center}
\end{table}

チューニングには、あるFEチップにおける全ピクセルのTresholdと任意の値に対するToTを調整するためのglobalチューニングと各ピクセルごとの値を目標値に近づけるlocalチューニングがある。はじめに、globalチューニングを行い、\fref{fig:analog}におけるIF DACおよび$5\ \si{bit}$のGDAC(\textbf{G}lobal \textbf{DAC})の値を調整し全ピクセルのThresholdまたはToTを大まかに目標値に合わせる。この段階では、全ピクセルから得られるThreshold分布およびToT分布の分散は大きいため、localチューニングにより\fref{fig:analog}におけるFDACおよび$7\ \si{bit}$のTDAC (\textbf{T}rim \textbf{DAC})の値を調整し、各ピクセルが返す値を目標値にさらに近づける。
%ToTの値はThresholdに依存するため、Thresholdのチューニングの前後においてToTが変わってしまう。また、同様にToTチューニングの後はThreshold値が変化してしまう。この影響は、チューニングを繰り返すことで小さくなり、

チューニングの後、ThresholdスキャンやToTスキャンを行い各ピクセルにおける値を測定する。ThresholdスキャンおよびToTスキャンの方法を以下に述べる。

%------------------------------------------------------------------------------------------------------------------------
\subsubsection{Thresholdスキャン}
\label{sec:thresholdscan}
%------------------------------------------------------------------------------------------------------------------------
Thresholdスキャンでは、各ピクセルに試験電荷を入射しThresholdとノイズを測定する。試験電荷を増加させつつ検出効率を測定し、\fref{fig:threshold}に示すような分布を作成する。この分布はS字を描くため、\textbf{Sカーブ}と呼ばれている。\fref{fig:threshold}中の青線はSカーブのフィッティングであり、\eref{eq:gosakannsuu}のような誤差関数を用いて定義される。
\begin{equation}
  \label{eq:gosakannsuu}
  f(x)=0.5\times\left[ 2-\mathrm{erfc}\left( \frac{x-Q_\mathrm{threshold}}{\sigma \times \sqrt{2}} \right)  \right]
\end{equation}
Sカーブにおいて、検出効率が$50\%$となる試験電荷の値をThresholdと定義し、検出効率が$16.5\%$と$83.5\%$となる試験電荷の幅の半値(\eref{eq:gosakannsuu}の$\sigma$)をノイズと定義する。

\begin{figure}[tbp]
  \centering
  \includegraphics[height=7cm,keepaspectratio]{threshold.png}
  \caption[検出効率と試験電荷の関係]{検出効率と試験電荷の関係 \cite{calibnoise}。黒点はある試験電荷に対する応答率の測定点であり、青線は\eref{eq:gosakannsuu}によるフィッティング結果を表す。検出効率が$50\%$となる試験電荷の値がThresholdであり、検出効率が$16.5\%$と$83.5\%$となる試験電荷の幅の半値がノイズである。}
  \label{fig:threshold}
\end{figure}


%------------------------------------------------------------------------------------------------------------------------
\subsubsection{ToTスキャン}
\label{sec:totscan}
%------------------------------------------------------------------------------------------------------------------------
ToTスキャンでは、一定の試験電荷を各ピクセルに100回入射させ、その試験電荷に対するToTの値の測定を行う。各ピクセルから得られるToTの値はデジタル値であるため整数値であるが、100回のスキャンの平均値をある試験電荷に対するToTとするため、この値は小数値を取り得る。


%------------------------------------------------------------------------------------------------------------------------
\subsection{電荷較正}
\label{sec:calibration}
%------------------------------------------------------------------------------------------------------------------------
\ref{sec:ASIC}節で示した様に、原理的にはToTと荷電粒子がシリコンセンサーに落とす電荷量$Q$は線形関係になると予想される。しかし、実際にはタイムウォーク等の二次的な効果を受け、線形関係ではなくなってしまう。\fref{fig:calibnijikouka}に、タイムウォークの影響が大きくなる小さい電荷量について、パルスの立ち上がり点をずらした際のToTの変化の様子を示す。タイムウォークの影響が大きくなる電荷量について、パルスの立ち上がり点がずれるとThresholdを超えるクロックウィンドウが1つ後ろにずれてしまうことが発生しやすくなり、小さいToTを出力する割合が増える。そのため、小さい電荷量におけるToTスキャンでは、原理的に予想されるToTと比べて平均的なToTが小さくなる。さらに、タイムウォークによりパルスの立ち上がりの傾きが電荷量によって異なることから、小さい電荷量においてToTと電荷量の関係は線形ではなくなると考えられる。

このような二次的な効果を含めた電荷較正式は、\eref{eq:calibration}のように表される。
\begin{equation}
  \label{eq:calibration}
  \mathrm{ToT} = p_0\frac{p_1 + Q}{p_2 + Q}
\end{equation}
\eref{eq:calibration}に示した3つのパラメータを求めるために、\fref{fig:analog}に示したアナログ回路の$V_\mathrm{cal}$の値を変えることにより電荷量を変化させつつ試験電荷を入射し、ToTの較正を行う。電荷較正は各ピクセルに対してパラメータを求めるのではなく、FEチップごとに一律の値を用いる。そのため、ToTスキャンから得られたToTの全ピクセルの平均値を用いてFEチップに対する電荷較正を行う。

%\begin{figure}[tbp]
%  \begin{minipage}[b]{0.5\linewidth}
%    \centering
%    \includegraphics[keepaspectratio, scale=0.6]{calibeq1.png}
%  \end{minipage}
%  \begin{minipage}[b]{0.5\linewidth}
%    \centering
%    \includegraphics[keepaspectratio, scale=0.6]{calibeq2.png}
%  \end{minipage}
%  \caption[ToTと荷電粒子がシリコンセンサーに落とす電荷量$Q$の概念図]{ToTと荷電粒子がシリコンセンサーに落とす電荷量$Q$の概念図。図中の青線と赤線はある電荷量$(Q_1>Q_2)$がToTに変換される概念図である。左図はToTと電荷量$Q$の関係が理想的に線形になる場合で、右図はタイムウォーク等の二次的な効果を受けた場合の図である。図中上半分の三角波はアナログ回路のフィードバック回路にて整形・増幅された信号であり、その信号がThreshold値を超える時間であるToTに変換される。三角波の立ち上がりはタイムウォークによって変化するが、立ち下がりはフィードバック回路によって制御されるため、電荷量に依存せず一定の傾きである。}
%  \label{fig:calibnijikouka}
%\end{figure}

\begin{figure}[tbp]
  \centering
  \includegraphics[height=5cm,keepaspectratio]{timewalklowtot.png}
  \caption[同じ電荷量のアナログ信号においてパルスの立ち上がり点をずらした際のToTの変化]{同じ電荷量のアナログ信号においてパルスの立ち上がり点をずらした際のToTの変化の様子。タイムウォークの影響が大きくなる電荷量に対しては、小さいToTを出力することが多くなる。}
  \label{fig:calibnijikouka}
\end{figure}

FE-I3を用いている現行ピクセル検出器は16bitのToTを持つのに対して、FE-I4を用いているIBLについてはToTの出力が4bitと少ない。このような理由から、Run3からはIBLは\eref{eq:calibration}によるフィッティングは行わず、ToTの値と試験電荷の値の対応情報を持つルックアップテーブルを用いて較正を行う。本研究では\eref{eq:calibration}を用いた電荷較正手法について取り扱うため、以下ではFE-I3を用いている現行ピクセル検出器の電荷較正のみについて述べる。

%------------------------------------------------------------------------------------------------------------------------
\section{電荷較正結果の履歴}
\label{sec:probrem}
%------------------------------------------------------------------------------------------------------------------------
Thresholdのチューニングや電荷較正を行った後、それらに関する情報はCERNに設置されているデータベース\cite{pixeldb}に保存する必要がある。このデータベースでは電荷較正に関する情報や検出器の配置や温度等のDCS(Data Control System)情報、さらにトリガー情報等を保管する。これらの情報は、測定におけるイベント選別やモンテカルロシミュレーションのためのイベント作成等に用いられる。

ThresholdスキャンおよびToTスキャンは各ピクセルごとの値を出力するが、データベースへはあるFEチップにおけるThresholdの平均値およびToTの平均値を用いた電荷較正式(\ref{eq:calibration})のパラメータのみを登録する。また、各ピクセルの大部分は\tref{tab:asicsiyou}に示した構造をしているが、FEチップの境界付近では不感領域をなるべく少なくするために、構造の異なったピクセルを配置する。FE-I3の境界付近におけるピクセルの構造を\fref{fig:pixeltypes}に示す。\tref{tab:asicsiyou}に示した通常のピクセルのことをnormalピクセル(図中の青の領域)と呼び、黄色の領域のピクセルをlongピクセル、赤色の領域をgangedピクセルと呼ぶ。
normalピクセルの大きさは$50\times 400\ \si{\micro m^2}$であるのに対して、longピクセルは$50\times 600\ \si{\micro m^2}$であり、長方形の長辺の長さがnormalピクセルの1.5倍となっている。また、gangedピクセルはnormalピクセル2つをワイヤーで接続した構造をしている。このような構造の違いから、ノイズ等の特性が異なるため、データベースへはそれぞれの値をアップロードする。

\begin{figure}[tbp]
  \begin{minipage}[1]{0.5\linewidth}
    \centering
    \includegraphics[keepaspectratio, scale=0.9]{FEI3zentaizu.png}
  \end{minipage}
  \begin{minipage}[1]{0.5\linewidth}
    \centering
    \includegraphics[keepaspectratio, scale=0.4]{pixeltypes2.png}
  \end{minipage}
  \caption[ピクセルモジュール全体図とFEチップ境界付近のピクセルタイプ]{ピクセルモジュール全体の概念図(左図)のFEチップ境界付近のピクセルタイプ(右図) \cite{pixeltypes}。左図はピクセルモジュール全体の概念図を表し、右図は左図の赤枠の領域を拡大したものである。FE-I3は$160\times18$ [行$\times$列]のピクセルを持ち、$2\times8$ [行$\times$列]のFE-I3を並べて1つのピクセルモジュールを構成する。FE-I3の1列目および18列目がlongピクセル、$154,\ 156,\ 158,\ 160$列目がgangedピクセルと定義される。2つのgangedピクセルの間にはinter-gangedピクセルというピクセルが存在するが、ノイズ等の特性はnormalピクセルと同等の値を持つ。}
  \label{fig:pixeltypes}
\end{figure}

%\begin{figure}[tbp]
%  \centering
%  \includegraphics[height=6cm,keepaspectratio]{pixeltypes2.png}
%  \caption[ピクセルモジュールのFEチップ境界付近のピクセルタイプ]{ピクセルモジュールのFEチップ境界付近のピクセルタイプ \cite{pixeltypes}。FE-I3は$160\times18$ [行$\times$列]のピクセルを持ち、$2\times8$ [行$\times$列]のFE-I3を並べて1つのピクセルモジュールを構成する。FE-I3の1列目および18列目がlongピクセル、$154,\ 156,\ 158,\ 160$列目がgangedピクセルと定義される。2つのgangedピクセルの間にはinter-gangedピクセルというピクセルが存在するが、ノイズ等の特性はnormalピクセルと同等の値を持つ。}
%  \label{fig:pixeltypes}
%\end{figure}

データベースに登録する情報を以下に示す。
\begin{itemize}
  \item Thresholdの平均値
  \item Thresholdの分散
  \item Thresholdのノイズ
  \item In-time threshold
  \item 電荷較正式(\ref{eq:calibration})における3つのパラメータ
  \item 電荷較正におけるフィッティングの誤差
\end{itemize}

これらのパラメータをデータベースに登録するために、電荷較正データを用いて各値を1つのファイルにまとめる必要がある。しかし、ピクセルモジュールの電荷較正では正しく結果を出力しない場合があるため、電荷較正結果を適切な値に補完する必要がある。次章において、電荷較正の際に生じる問題とその補完方法について説明する。



\newpage

\chapter{電荷補正の最適化}

\section{これまでの補正方法}

\section{電荷較正の補正}

\section{データが欠陥した際の補正}

\section{本章のまとめ}

\newpage

%----------------------------------------------------------------------------
\chapter{次世代ピクセル検出器の量産}
\label{sec:singatapixel-devel}
%----------------------------------------------------------------------------
〜と同時に、HL-LHCアップグレードに向けた内部飛跡検出器の総入れ替えのため、次世代ピクセル検出器の開発が進められている。現在、ITkに搭載するピクセル検出器量産の各組み立て工程における試験やそのシステムの確立のため、試作器を用いたデモンストレーションが行われている。

日本では新型器量産の際に約$2000$個のモジュールを生産する予定である。新型器の量産の際に、効率の良い量産と統合されたモジュール選定を行うため、品質試験結果を統合管理するシステムの開発が必要となる。


%----------------------------------------------------------------------------
\section{次世代ピクセル検出器の組み立て部品}
\label{sec:component}
%----------------------------------------------------------------------------
量産工程は、各組み立て機関に届いたセンサーとASICから作られるベアモジュールとフレキシブル基板の接着から始まる。本節では各部品の詳細について説明する。


%----------------------------------------------------------------------------
\subsection{ベアモジュール}
\label{sec:bare}
%----------------------------------------------------------------------------

ベアモジュールはセンサーとASICをバンプ接合することにより作られる。クアッドモジュールではセンサー1枚に対してASIC4枚、トリプレットモジュールではセンサー1枚に対してASIC1枚から構成される。ベアモジュールは通過する粒子を検出する、モジュールの中でも重要な部品である。センサーを通過した荷電粒子は電子・ホール対を生成し、それにより得られる信号をASICを用いて増幅・整形を行う。\fref{fig:bare}にベアモジュールの全体図を示す。
\begin{figure}[tbp]
  \centering
  \includegraphics[height=5cm,keepaspectratio]{bare.png}
  \caption[ベアモジュール]{ベアモジュールの全体図。センサー側から見たものであり、左右にASICがはみ出している。これはフレキシブル基板につながるワイヤーのためのパッドが存在する部分である。}
  \label{fig:bare}
\end{figure}

%現在行われている試作器の組み立てではRD53AというASICを用いている。


%----------------------------------------------------------------------------
\subsection{フレキシブル基板}
\label{sec:flex}
%----------------------------------------------------------------------------

フレキシブル基板はセンサーの裏側(\fref{fig:bare}ので見えている面)に接着され、

\begin{figure}[tbp]
  \centering
  \includegraphics[height=6cm,keepaspectratio]{flex.png}
  \caption[フレックス基板]{フレックス基板の全体図。}
  \label{fig:flex}
\end{figure}

フレキシブル基板は、以下の3つの役割を持つ。
\begin{itemize}
  \item ASICからの信号輸送  \\
  センサーから得られた信号はASICで増幅・整形され、フレキシブル基板に送られてくる。フレキシブル基板は送られてきた信号を後段のPCへ送る。
  \item 電源の供給 \\
  外部からの電源を、センサーとASICに供給する。センサーには、空乏領域を増加させるために$100\ \si{V}$程度のHV(\textbf{H}igh \textbf{V}oltage)をかける。ASICには、電源供給のために$5.6\ \si{V}$程度のLV(\textbf{L}ow \textbf{V}oltage)
  \item モジュールの制御システム(DCS: \textbf{D}etector \textbf{C}ontrol \textbf{S}ystem) \\
  モジュールの温度測定のために2つのNTC(\textbf{N}egative \textbf{T}emperature \textbf{C}oefficient)が配置されている。
\end{itemize}

%----------------------------------------------------------------------------
\subsection{モジュールキャリア}
\label{sec:carrier}
%----------------------------------------------------------------------------

モジュールキャリアはモジュールの運搬の際や品質試験を行う際に、モジュールを保護用の容器である。組み立てられたモジュールはASICとフレックス基板を繋ぐワイヤー部やセンサーの部分等が剥き出しになっているため、そのままの状態で品質試験を行うのはモジュール破損のリスクを伴う。モジュールキャリアでモジュールを保護することにより、安全に品質試験を行うことができる。

また、モジュールキャリアの別の役割として、モジュール周囲の湿度環境を一定に保つことが挙げられる。運転時に想定される最低温度は$-45\ \si{\degreeCelsius}$のため、品質管理試験ではペルチェ素子を用いた温度制御装置\footnote{KEKにおける次世代ピクセル検出器の量産では、東工大を中心に開発している温度制御システムを用いる。}を用いて$-45\ \si{\degreeCelsius}$までモジュールの周囲温度を下げる。その際、モジュールへの結露を防ぐための結露を防ぐためキャリア内に乾燥窒素ガスを流し込むことで氷点下においてもモジュールへの結露を防いでいる。

%----------------------------------------------------------------------------
\section{次世代ピクセル検出器の組み立て工程}
\label{sec:assemble}
%----------------------------------------------------------------------------
\begin{figure}[tbp]
  \centering
  \includegraphics[height=7cm,keepaspectratio]{module_flow.png}
  \caption[ATLAS検出器]{ATLAS検出器の全体図 \cite{studyofID} }
  \label{fig:assemble}
\end{figure}


次世代ピクセル検出器の組み立て工程を\fref{fig:assemble}に示す。組み立て工程ではフレキシブル基板とベアモジュールの接着から始まり、ワイヤー配線、パリレン高分子によるコーティング、ワイヤー保護を行いピクセルモジュールが完成する。その後、温度サイクル試験および低温耐久試験において、運転時に想定される温度環境において組み立てたモジュールが運用できるかの試験を行う。本節では、組み立て工程、およびモジュールの温度耐久についての試験についての説明を示す。

\subsubsection*{ベアモジュール・フレキシブル基板の接合}

モジュールの組み立て工程は、組み立て機関に輸送されたベアモジュールとフレキシブル基板の接合から始まる。輸送された各部品の外観検査等の受け取り時の試験を行った後、ベアモジュールとフレキシブル基板の接合を行う。接合の後、モジュールの外観検査・平坦性測定・質量測定を行う。

\subsubsection*{ワイヤー配線}

フレキシブル基板からセンサーおよびASICへの電圧の供給や、ASICからの信号を読み出すため、フレキシブル基板とセンサーおよびASIC間をワイヤーで接続する。

\subsubsection*{パリレンコーティング}

モジュールのセンサーとASICの端の部分での放電を防ぐこと、湿気や化学物質からの保護を目的としてパリレンコーティングを行う。パリレンはパラキシリレン系ポリマーの略である。パリレンは結晶性が高く絶縁耐力に優れ、周波数に依存せず低い誘電率・誘電正接特性を持っており、湿気や腐食性ガスへの耐性も併せ持つ。


\subsubsection*{ワイヤー保護}

ワイヤーは直径$25\ \si{\micro m}$のため、非常に細い。

\subsubsection*{温度サイクル}

組み立てたモジュールに対して、ITk実装後にされる特異的な温度変化のサイクルを行い、その後もモジュールが正常な応答をするか試験をする。温度変化の際、フレキシブル基板にたわみが生じ、それが原因でバンプ接合部に剥がれが生じてしまうことがある。このようの温度サイクルによるモジュールの損傷がないことを確認する必要がある。

温度サイクルは、動作温度範囲$-45\ \si{\degreeCelsius}<T<40\ \si{\degreeCelsius}$で$100$サイクル以上、$-55\ \si{\degreeCelsius}<T<60\ \si{\degreeCelsius}$を$1$サイクルである。

\subsubsection*{低温耐久試験}

ITk運転におけるピクセル検出器の周囲温度は$-15\ \si{\degreeCelsius}<T<0\ \si{\degreeCelsius}$である。組み立てたモジュールが低温環境下において長時間正常に動作することを確認する試験が低温耐久試験である。
低温耐久試験では、温度制御筐体を用いてモジュールの周囲温度を$-15\ \si{\degreeCelsius}$に保ちつつASICの回路読み出し試験を行う。読み出し試験は1時間に1度行われる。

長時間放置しつつ読み出し試験を行うため、インターロックシステム、機器の遠隔制御、温度制御筐体の遠隔監視等の技術が必要となる。


%----------------------------------------------------------------------------
\section{品質試験}
\label{sec:QCtest}
%----------------------------------------------------------------------------




\subsection{}

%----------------------------------------------------------------------------
\section{量産における試験結果管理}
\label{sec:}
%----------------------------------------------------------------------------



\newpage

%------------------------------------------------------------------------------------------------------------------------
\chapter{モジュール量産におけるデータベースシステム}
\label{sec:chap6}
%------------------------------------------------------------------------------------------------------------------------

前章に示したように、次世代モジュールの量産の時にデータを適切に管理するためにデータベースシステムを開発している。

%測定と結果の一貫性と比較可能性を確保することが重要になる。
本章ではデータベースシステムの全体像について説明する。

%\begin{lstlisting}[caption=hoge,label=fuga, language=C++]
%#include<stdio.h>
%int main(){
%   printf("Hello world!");
%}
%\end{lstlisting}

%------------------------------------------------------------------------------------------------------------------------
\section{量産に用いるデータベースの概要}
\label{sec:DBforMasspro}
%------------------------------------------------------------------------------------------------------------------------

量産に用いるデータベースは、ITkの製造に関する情報を全てを記録するために開発されている中央データベースと、各研究機関でモジュール量産を管理するためのローカルデータベースがある。中央データベースとローカルデータベースの設置位置を\fref{fig:sekainohate}に示す。
本節ではそれぞれについての概要を示す。
\begin{figure}[tbp]
  \centering
  \includegraphics[height=5.5cm,keepaspectratio]{sekainohate.png}
  \caption[中央データベースとローカルデータベースの設置位置]{中央データベースとローカルデータベースの設置位置。ローカルデータベースは全てのモジュール組み立て機関に設置される。}
  \label{fig:sekainohate}
\end{figure}
%\begin{figure}[tbp]
%  \centering
%  \includegraphics[height=7cm,keepaspectratio]{ryousanDB.png}
%  \caption[中央データベースとローカルデータベース]{中央データベースとローカルデータベース}
%  \label{fig:ryousanDB}
%\end{figure}

%------------------------------------------------------------------------------------------------------------------------
\subsection{中央データベース}
\label{sec:ITkPD}
%------------------------------------------------------------------------------------------------------------------------

中央データベースはITkに実装するためのピクセルモジュール・ストリップ検出器の量産についてのデータを管理することを目的として開発されている。ユニコーン大学が中心に開発を行っており、チェコにデータベースサーバーが設置されている。

ITkに実装される検出器とその構成部品は$50$万個程度であり、中央データベースではそれら全ての量産についてのデータを管理する。量産の組み立て工程で行われる試験結果は全てを記録するのではなく、ある品質管理試験における良い結果を一つのみを保存しておく。

中央データベースに保存されたデータは、ITkに配置するピクセルモジュールの選別に用いられる。$|\eta|$が小さい領域は通過する粒子の密度が高くなることが想定されるため、高い放射線量が予想される。そのため、ITkに搭載する際にはなるべく品質の良いモジュールを搭載する予定である。また、品質の悪いモジュールを配置する領域が固まると、その領域を通過する飛跡の再構成が難しくなり、物理解析の際に不良データと判別されてしまうことがある。そこで、品質の悪いモジュールの領域が固まらないように配置する必要
がある。この選別に用いる参考値として、中央データベースに保存されているデータを用いる予定である。

また、運転前後で検出器に損傷が起きていないかを確認するためにもこれらの値を用いる。運転前後の読み出し試験やその他の試験の結果の比較を行うことにより、センサーへの放射線損傷や不良ピクセルの推移等を確認することができる。
ITkは約$10$年間の運転が予定されており、少なくとも$20$年間は検出器についてのデータが利用できるようにしておく予定である。



%------------------------------------------------------------------------------------------------------------------------
\subsection{ローカルデータベース}
\label{sec:LocalDB}
%------------------------------------------------------------------------------------------------------------------------

ローカルデータベースは各研究機関に設置され、モジュールの組み立て工程における品質試験結果を管理するためのシステムである。ローカルデータベースは各研究機関におけるモジュールの品質試験結果を全て記録しておくという点で中央データベースとは異なる。

先述したように、モジュールの組み立て工程およびそれぞれの工程の際に行われる品質試験の数は非常に多く、それぞれのモジュールに対して適切に試験を管理する必要がある。モジュールの量産は各研究機関において$\mathcal{O}(100)\sim \mathcal{O}(1000)$個であり、1つのモジュールについて$30$項目程度の品質試験を行うため、$\mathcal{O}(10000)$個の試験結果を適切に管理する必要がある。各組み立て機関において、適切にデータを管理し、かつピクセルモジュールの量産を円滑に行うことができるようにサポートするために\textbf{ローカルデータベース}というシステムを開発している。ローカルデータベースに対する要求は以下のようなものである。
\begin{itemize}
  \item 検出器情報や試験結果の情報を他研究機関との整合性を保ちつつ管理すること。
  \item ピクセルモジュールの量産を円滑に行うことができるようサポートすること。
  \item 中央のデータベースにモジュール量産に関わるデータを同期すること。
\end{itemize}

以上の目標を達成するために、東工大を中心にローカルデータベースの開発が進められている。次節において、ローカルデータベースの構造と、先行研究における開発項目について示す。

%------------------------------------------------------------------------------------------------------------------------
\section{ローカルデータベースの構造}
\label{sec:AboutLocalDB}
%------------------------------------------------------------------------------------------------------------------------
ローカルデータベースの構造の全体像を\fref{fig:localdbsystem}に示す。ローカルデータベースは品質試験結果を保有しておくためのデータベースであるMongoDBと、データベースに保存されたデータを操作および閲覧するためのウェブアプリケーションから構成される。MongoDBへの試験結果の登録は、読み出し試験については専用のソフトウェアであるYARRを用いて行い、それ以外の品質試験については大阪大学および都立大学によって開発が進められている品質試験結果登録用GUIを用いて行う。
本節では、MongoDBとウェブアプリケーションについて説明する。


\begin{figure}[tbp]
  \centering
  \includegraphics[height=7cm,keepaspectratio]{localdbsystem.png}
  \caption[ローカルデータベースの全体像]{ローカルデータベースの全体像。ローカルデータベースはMongoDBとウェブアプリケーションから構成され、品質試験結果を管理するための複合的な機能を持つ。}
  \label{fig:localdbsystem}
\end{figure}

%------------------------------------------------------------------------------------------------------------------------
\subsection{MongoDB\cite{mongo}}
\label{sec:mongo}
%------------------------------------------------------------------------------------------------------------------------

MongoDBはアプリケーションの開発や拡張を簡単に行うことができるように設計されている、オープンソースのドキュメントデータベースである。ドキュメントデータベースは、非リレーショナルデータベース(NoSQL)の一種であり、1つのデータをドキュメントと呼び、具体的にはデータをJSON\footnote{JSONとは\textbf{J}ava\textbf{S}cript \textbf{O}bject \textbf{N}otation の略で、データ記述言語の一種である。あるkeyとvalueを対応させることにより、データを取り出すことができる。ウェブアプリケーションでデータを転送する場合に使われることが多い。}のようなドキュメントとして保存する。MongoDBでは、\textbf{BSON}という、JSONに非常によく似た形式のデータを扱うが、binary表記したデータを持つためJSONよりも多くのデータの型を保存することができる。

MongoDBでは、データベース、コレクション、ドキュメント(オブジェクト)の概念を用いてデータを管理する。各ドキュメントはコレクションという枠の中に格納され、さらに各コレクションはデータベースによって包括される。
各コレクション間の関係を自由に定義することができ、データベースが構造化し、階層的な性質を持たせてデータを管理することができる。
%\fref{fig:mongoimage}にMongoDBのデータを保管するための概念図を示す。

%\begin{figure}[tbp]
%  \centering
%  \includegraphics[height=7cm,keepaspectratio]{mongoimage.png}
%  \caption[MongoDBの全体像]{MongoDBの全体像}
%  \label{fig:localdb-collection}
%\end{figure}

各ドキュメントは\textbf{オブジェクトID}という$12\ \si{bite}$のIDによって管理される。オブジェクトIDはデータ生成時の時間情報によって決まる値($4\ \si{bite}$)、データベースマシンによってランダムに生成される値($5\ \si{bite}$)、データベースマシンのカウンターによって生成される値($3\ \si{bite}$)により自動生成される。そのため、複数のドキュメントが同一のオブジェクトIDを持つことはなく、これにより識別を行うことができる。また、オブジェクトIDを用いることにより、異なるコレクションにおけるドキュメント間の関連付けを行い、データベース内の構造を柔軟に設計することができる。

\begin{lstlisting}[caption=MongoDBのドキュメントの例,label=mongodocument, language=Python]
{
  {
    "_id": ObjectId("6038c960b9a87924947df638"),
    "year": 2015,
    "title": "The Big New Movie",
    "info": {
      "plot": "Nothing happens at all.",
      "rating": 0
    }
  }
}
\end{lstlisting}


ローカルデータベースシステムの開発において、MongoDBを使う利点を以下に示す。
\begin{itemize}
  \item スキーマレスで、ドキュメント構造を動的に変更することができる。 \\
  品質管理試験は、試験項目により保存するパラメータが異なる。格納形式に柔軟性のあるNoSQLのMongoDBを用いることにより、同一コレクションに異なる形式の試験結果を保管しておくことができる。また、現在は試作器を用いて、次世代器の品質試験に向けた実験装置の準備や試験のレビューを行っている。そのため、最終的にデータベースに残る結果が変わることがある。この際、NoSQLのMongoDBでは変更点が最小に抑えることができるので、開発スピードを早くできることが多い。
%  \item JSON形式でデータを保持するため整形が容易である。\\
%  We love JSON format.
  \item ObjectIdにより、中央データベースとの整合性を保ちやすい。 \\
  中央データベースで管理するデータは$12\ \si{byte}$のIDを保持している。そのため、中央データベースにおけるIDとローカルデータベースのMongoDBにおけるObjectIdを関連付けて管理することが可能になる。
\end{itemize}

%%------------------------------------------------------------------------------------------------------------------------
%\subsection{LocalDBのデータ管理構造}
%\label{sec:localDB-structure}
%%------------------------------------------------------------------------------------------------------------------------
これらの利点から、MongoDBを用いてローカルデータベースの開発を行っている。ローカルデータベースにおけるモジュール情報および品質管理に用いるコレクションを\tref{tab:local-collection}に示す。ローカルデータベースにおいて、\textbf{localdb}と\textbf{localdbtool}の2つのデータベースを準備している。localdbはピクセルモジュール情報および品質管理試験結果等の各組み立て機関から中央データベースに共有する情報を保有し、localdbtoolはユーザー情報、中央データベースからダウンロードした組み立て機関のリストやピクセルモジュールの構成要素等の中央データベースに共有しない情報や組み立て機関に依存しない情報を保有する。
各コレクション間の関係を\fref{fig:localdb-collection}に示す。このように各コレクションに保存する情報を関連付けて管理することにより、品質試験結果やそれに関する情報をデータベースから取り出しやすくなる。

\begin{table}[tbp]
  \begin{center}
    \caption[ローカルデータベースのコレクション]{ローカルデータベースのコレクション一覧。}
    \label{tab:local-collection}
    \begin{tabular}{|l||l|l|}
    \hline
      データベース名 & コレクション名 & 保存情報 \\
    \bhline{1.5pt}
      \multirow{11}{*}{localdb}
       & component & モジュール情報、FE チップ情報 \\
       & childParentRelation & FE チップとモジュールの関係性 \\
       & testRun & 読み出し試験結果 \\
       & componentTestRun & component と testRun の関係性 \\
       & user & 読み出し試験実施者 \\
       & institute & 読み出し試験実施場所 \\
       & comments & 部品、試験結果についてのコメント情報 \\
       & QC.module.status & 各モジュールに対する組み立て工程及び選択された試験結果 \\
       & QC.result & 品質試験結果 \\
       & QC.prop.status & ワイヤー配線の後に決まるモジュール特性の書き換え情報 \\
       & QC.module.prop & モジュール特性の情報 \\
    \hline
      \multirow{5}{*}{localdbtools}
       & QC.status & 組み立て工程及び試験項目 \\
       & QC.module.types & モジュールの構成部品 \\
       & viewer.user & ローカルデータベースのユーザ情報 \\
       & viewer.query & 読み出し結果キーワード、検索機能実行時に使用 \\
       & viewer.tag.docs & モジュールや試験結果に付けるタグの情報 \\
    \hline
    \end{tabular}
  \end{center}
\end{table}

\begin{figure}[tbp]
  \centering
  \includegraphics[height=12cm,keepaspectratio]{localdbcollection.png}
  \caption[ローカルデータベースの構造]{ローカルデータベースの構造。それぞれの四角はコレクションを表しており、緑色は使用ユーザーに関する情報、赤色は部品についての情報、青色はYARRを用いて登録した読み出し試験結果、紫は品質試験結果登録用GUIを用いて登録した結果についてのものである。また、灰色はモジュールの組み立て工程を管理するためのコレクションであり、黄色は読み出し試験データや画像データを管理するファイルシステムに関するコレクションである。直線はオブジェクトIDによるドキュメント間のリンクを示している。}
  \label{fig:localdb-collection}
\end{figure}



%------------------------------------------------------------------------------------------------------------------------
\subsection{ウェブアプリケーション}
\label{sec:flask}
%------------------------------------------------------------------------------------------------------------------------

各研究機関において、ローカルデータベースを使用するために、試験者がデータベースに保存されている品質試験結果を閲覧および管理を簡単にできる必要がある。
しかし、MongoDBはデータ構造が柔軟であり、使用用途に基づき多様なデータベース構造が考えられるため、データベース内のデータを表示・処理するインターフェースは提供されておらず、必要に応じて適宜インターフェースを開発する必要がある。
ローカルデータベースシステムでは、多様な利用者がデータベース内のデータを閲覧、操作を実現できることを目標として、PythonのウェブアプリケーションフレームワークであるFlaskを導入している。

ウェブアプリケーションの処理を\fref{fig:webapp}に示す。利用者が見るウェブブラウザのインターフェイスはhtmlで書き、ボタンを押したらPythonのバックエンド側にパラメータが送信される。受信したパラメータをもとにFlaskで処理を返し、html側で表示を行う。Flaskが受け取った処理を行う際、MongoDBに保存するデータを操作するために、PythonのパッケージであるPyMongoを用いている。また、ウェブブラウザにおいて、動的な処理を行う際にはJavaScriptを利用している。この流れにより、ウェブブラウザから、データベースに保存されているデータを簡単に取り扱うことができるように設計している。

\begin{figure}[tbp]
  \centering
  \includegraphics[height=4cm,keepaspectratio]{localdbwebapp.png}
  \caption[ウェブアプリケーションの処理の概念図]{ウェブアプリケーションの処理の概念図。図中の数字は処理の順序を表し、ウェブブラウザから送信した信号に基づきウェブアプリケーション内で処理を行う。}
  \label{fig:webapp}
\end{figure}
%\begin{figure}[tbp]
%  \centering
%  \includegraphics[height=7cm,keepaspectratio]{localdbwebapp.png}
%  \caption[ローカルデータベースシステムの全体像]{ローカルデータベースシステムの全体像}
%  \label{fig:webbrowser}
%\end{figure}

%------------------------------------------------------------------------------------------------------------------------
\section{モジュールの品質試験に必要な開発項目}
\label{sec:okubottan}
%------------------------------------------------------------------------------------------------------------------------

東工大を中心に開発が進められており、これまでは読み出し試験を中心に開発が進められている。モジュールの品質管理に必要な開発項目および開発状況を以下に示す。
\begin{itemize}
  \item[1.] データベース構造の設計 \\
  \fref{fig:localdb-collection}に示したように、MongoDBにおけるコレクションを定義し、データを取り出しやすく工夫している。読み出し試験についてのデータベース構造は先行研究\cite{kubotan,kimu}によって定義され、非読み出し試験およびモジュールの組み立て工程を管理するためのデータベース構造は先行研究\cite{oku}を行った奥山氏と私が開発を行った。
  \item[2.] 試験結果管理機能
  \begin{itemize}
    \item[2-1.] 読み出し試験 \\
    読み出し試験の結果の閲覧および解析機能は先行研究\cite{oku,kubotan}によって開発が行われた。読み出し試験はピクセルの読み出し性能の不良判定するのに重要な試験であり、各組み立て工程で品質に変化がないこと、あるいは不良ピクセルが発生した際にどの工程で問題があったかを発見することが重要である。本研究では、各組み立て工程間で評価できる機能を追加した。これについて\ref{sec:hyouji}に示す。
    \item[2-2.] 非読み出し試験 \\
    非読み出し試験の閲覧機能はこれまで未開発であり、本研究においてこの機能を実装した。閲覧機能について、\ref{sec:hyouji}に示す。
  \end{itemize}
  \item[3.] モジュール組み立て工程管理機能 \\
  各組み立て機関で$\mathcal{O}(100)$〜$\mathcal{O}(1000)$個のモジュールを量産するため、それぞれのモジュールについて品質試験が適切に行われたこと、各工程で全ての結果が揃っていることを担保する必要がある。この機能は先行研究\cite{oku}によって開発が行われたが、モジュール特性の結果管理についてまだ未開発であった。これについて\ref{sec:kanri}に示す。
  \item[4.] 中央データベースとの同期機能 \\
  これまで、モジュールを中央データベースからダウンロードする機能、および読み出し試験の結果を中央データベースと同期する機能の開発が先行研究\cite{oku}によって行われてきた。しかし、モジュールを中央データベースへ登録、非読み出し試験の結果およびモジュールの組み立て工程を中央データベースと同期する機能については未開発であった。これについて、\ref{sec:upload-result}に示す。
\end{itemize}






\newpage

\chapter{試験結果データ管理システムの開発}

\section{ピクセル検出器情報の登録}

\section{試験結果の管理}

\section{試験結果のアップロード・ダウンロード}

\section{試験結果の評価}

\newpage

%------------------------------------------------------------------------------------------------------------------------
\chapter{まとめ}
\label{sec:chap8}
%------------------------------------------------------------------------------------------------------------------------

%------------------------------------------------------------------------------------------------------------------------
\section{結論}
\label{sec:ketsuron}
%------------------------------------------------------------------------------------------------------------------------

CERNにある世界最高エネルギーでの陽子衝突加速器LHC上の測定点の1つであるATLAS実験では、標準模型の精密測定や標準模型を超える物理現象の探索が行われている。LHCは2022年の3月に長期運転停止期間を終えてRun3として稼働を再開する。
ATLASピクセル検出器はこれまで2週間から1か月に1度の電荷較正を行っていた。電荷較正を行った後にそれぞれのパラメータが適切な値を保持していることを確認および欠損が含まれるデータの補完作業を行った後に、データベースにパラメータの登録を行う。データの確認および補完作業はこれまで担当者による手作業で行われていた。しかし、Run1から稼働しているピクセル検出器はこれまでよりも放射線損傷による影響がより大きくなることから、より頻繁に電荷較正およびその補完作業を行う必要があり、手作業による確認および補完作業は非常に労力が伴う。さらに、手作業による補正であることから、担当者によって補完方法の偏りが生じてしまう。そのため、電荷構成結果の適切な補完手法を確立し、自動で補完を行う解析ツールが必要である。

本研究では、電荷較正結果を確認し補完を行う解析ツールの開発を行った。電荷較正の際に発生しうる問題は2種類ある。1つ目の問題は、電荷較正を行う際に正しい試験電荷が生成できないことがある。この問題を検知するために電荷較正結果の新たな評価方法を導入し、問題のある試験電荷を順に取り除くアルゴリズムを開発した。2つ目の問題は、電荷較正結果に含まれるパラメータの欠損である。これまでの補完方法は最も近いFEチップから値をコピーするという方法であり、担当者により異なる値による補完を行ってしまうことがあるため、パラメータの最適な補完方法の評価を行った。その結果、Threshold値については同一FEチップにおける異なるピクセルタイプの平均、その他のパラメータについては異なるFEチップにおける同一ピクセルタイプの平均を用いることにより、より実際の値に近い値を再現できるという結果が得られた。この結果を利用し、電荷較正結果に含まれるパラメータの欠損を自動補完する解析ツールの開発を行った。開発した解析ツールを用いて2022年9月に行われた電荷較正データを用いて、Run3モンテカルロシミュレーションサンプル作成のための電荷較正結果の作成を行った。

また、統計数を増加させ新物理発見の感度を向上させるため、LHCでは2024年からHL-LHCへのアップグレードが計画されている。HL-LHCでは、陽子ビームのバンチに含まれる陽子数が増加するため、瞬間ルミノシティが現行LHCの$5$-$7$倍になり取得統計量の向上が期待される。そのため、1バンチあたりの信号数が増加するため、検出器には読み出し速度の高速化と、高い放射線耐性、およびイベントのパイルアップを防ぐために高細密化が要求される。陽子の衝突点から最も近い内部飛跡検出器\footnote{\ref{sec:InnerDetector}節で説明したように、内部飛跡検出器はIBL、ピクセル検出器、ストリップ検出器、遷移放射検出器から構成される。}は、要求性能を満たすために検出器の総入れ替えが予定されている。そのために、次世代ピクセルモジュールの大量生産が予定されており、各ピクセルモジュールに対して品質試験を行う。さらに、ATLASに搭載する際のモジュール選別や運転前後の性能比較のために、品質試験結果はチェコにある中央データベースに保管しておく必要がある。

本研究では、効率の良い量産と統合されたモジュール選定のために、品質試験結果の表示機能、品質試験結果の管理機能、および中央データベースとの同期機能の開発を行った。これらの機能をローカルデータベースに実装することにより、ピクセルモジュールの次世代器量産における品質試験結果管理に必要な機能の基本的な構成要素が全て揃った。しかし、品質試験結果を中央データベースへアップロードする機能について、読み出し試験結果のアップロード処理に約$160$秒必要という結果が得られた。処理時間改善のため、読み出し試験結果のアップロード処置を細分化し、処理時間を調査した。その結果、ローカルデータベースから試験結果を抽出処理を並列化することにより、処理時間が最大$50$秒程度削減できるということがわかった。


%------------------------------------------------------------------------------------------------------------------------
\section{今後の課題}
\label{sec:konngonokadai}
%------------------------------------------------------------------------------------------------------------------------

%------------------------------------------------------------------------------------------------------------------------
\subsection{電荷較正の自動補完ツール}
\label{sec:dennkahoseinokonngonokadai}
%------------------------------------------------------------------------------------------------------------------------

Run3においてピクセル検出器の電荷較正は10日に1度程度の頻度で行われる。この作業は1人の担当者により行われる予定であるが、担当者が体調不良等が原因で作業を行うことができなくなると別の人が作業を行うことになる。本研究において開発した電荷較正結果の補完ツールについて、補完方法の説明や使用方法の説明のためのドキュメント作成を行う必要がある。

%また、2021年9月に行われた電荷較正のデータを用いて電荷較正の補完を行った。その際に、電荷較正結果を確認すると、小さい試験電荷を生成した際に2つの構造が確認でき、ToT$=4$が得られた際に、較正式から得られる電荷と試験電荷の違いが20\% 程度あるということがわかった。このような違いがクラスタリングから得られる荷電粒子の通過位置測定に、どのような影響を及ぼすかをシミュレーション等を用いて正確に評価する必要がある。

%補完ツールはコマンドラインベース(CUI)にて実行するものである

%------------------------------------------------------------------------------------------------------------------------
\subsection{次世代ピクセルモジュール量産のためのデータベースシステム}
\label{sec:dbnokonngonokadai}
%------------------------------------------------------------------------------------------------------------------------

本研究では、読み出し試験以外の項目についての閲覧機能や同期機能等を開発することにより、次世代ピクセルモジュールの量産に必要な機能の基本的な部分が揃った。ローカルデータベースシステムは各組み立て機関に設置されるものであるため、国外の研究機関が使用できるようにユーザーサポートを行う必要がある。そのために、ソフトウェア使用のためのドキュメントの作成やグループチャットによる問合せ対応を行っている。このようなユーザーサポートを今後も継続していく必要がある。

また、前章で述べたように、ローカルデータベースから中央データベースへの品質試験結果のアップロード機能のように、処理時間が長くユーザーの使用満足度が非常に悪い部分が確認された。読み出し試験結果の閲覧機能についても、結果表示のために数分必要ということが確認されていることから、処理時間を短くするような工夫が必要である。

\newpage

%
%
\appendix
%
%------------------------------------------------------------------------------------------------------------------------
\chapter{非読み出し試験結果のブラウザ表示}
\label{sec:A}
%------------------------------------------------------------------------------------------------------------------------

%------------------------------------------------------------------------------------------------------------------------
\section{非読み出し試験結果のブラウザ表示}
\label{sec:A1}
%------------------------------------------------------------------------------------------------------------------------

%------------------------------------------------------------------------------------------------------------------------
\subsection{外観検査}
\label{sec:A11}
%------------------------------------------------------------------------------------------------------------------------


%------------------------------------------------------------------------------------------------------------------------
\section{ピクセル検出器の基本特性のブラウザ表示}
\label{sec:A2}
%------------------------------------------------------------------------------------------------------------------------

%\chapter{Appendix B}
増やしたいときはこちらにどうぞ。
%
\begin{thebibliography}{24}
\bibitem{soryuushi}
南部陽一郎・木下東一郎・牧二郎・中西襄・政池明,
”大学院 素粒子物理学1 素粒子の基本的性質”
講談社サイエンティフィック
\bibitem{oku}
奥山広貴, ”HL-LHC ATLAS ピクセル検出器量産時の品質試験に向けた データベースシステムの構築”,
\url{https://cernbox.cern.ch/index.php/s/IBO6BptySxPwD6L}, 東京工業大学大学院修士論文 (2021)
\bibitem{kubotan}
窪田ありさ, ”HL-LHC ATLAS 実験用新型ピクセル検出器の系統評価と 量産時に向けた試験管理システムの開発”,
\url{https://cernbox.cern.ch/index.php/s/BdhvSTAuuE5xHXt}, 東京工業大学大学院修士論文 (2020)
\bibitem{kimu}
Eunchong Kim, ”Development of DAQ test system and database for the HL-LHC ATLAS production Pixel detector",
\url{https://cernbox.cern.ch/index.php/s/ZyYwYnMxeCSoeZl}, 東京工業大学大学院修士論文(2019)
\bibitem{LHC}
The CERN accelerator complex, ”The CERN accelerator complex”, \url{https://cds.cern.ch/images/OPEN-PHO-ACCEL-2013-056-1}, CERN Document Server (最終更新: 2018年7月)
\bibitem{ATLAS}
Computer generated image of the whole ATLAS detector, "Computer generated image of the whole ATLAS detector", \url{https://cds.cern.ch/record/1095924}, CERN Document Server (最終更新: 2015年2月).
\bibitem{studyofID}
The ATLAS Collaboration, "Study of the material of the ATLAS inner detector for Run 2 of the LHC"(2017), \url{https://atlas.web.cern.ch/Atlas/GROUPS/PHYSICS/PAPERS/PERF-2015-07/}
\bibitem{calocalo}
Computer Generated image of the ATLAS calorimeter, "Computer Generated image of the ATLAS calorimeter", \url{https://cds.cern.ch/images/CERN-GE-0803015-01}, CERN Document Server (最終更新: 2018年6月).
\bibitem{hl-lhc}
CERN Accelerating science, “The HL-LHC project”, 2021年12月閲覧,
\url{https://hilumilhc.web.cern.ch/content/hl-lhc-project}
\bibitem{lhc-lumi}
Burkhard Schmidt, "The High-Luminosity upgrade of the LHC: Physics and Technology Challenges for the Accelerator and the Experiments" (2015),
\url{https://iopscience.iop.org/article/10.1088/1742-6596/706/2/022002/pdf}
\bibitem{crab}
細山 謙二, "超伝導クラブ空洞" (2008), \url{https://www.jstage.jst.go.jp/article/jcsj/43/4/43_4_132/_pdf}
\bibitem{itk}
The ATLAS Collaboration, "ATLAS Inner Tracker Pixel Detector Technical Design Report" (2018), \url{https://cds.cern.ch/record/2285585/files/ATLAS-TDR-030.pdf}
\bibitem{higgs}
The ATLAS Collaboration. “Combined measurements of Higgs boson production and decay using up to $80$ fb$^{-1}$ of proton–proton collision data at $\sqrt{s}=13$ TeV collected with the ATLAS experiment” (2018). CERN Document Server. \url{http://cdsweb.cern.ch/record/2629412/}
%\bibitem{kek}
%\url{https://atlas.kek.jp/main/movie/photos/physics/index.html}
%\bibitem{darkmatter}
%\url{https://home.cern/science/physics/dark-matter}
\bibitem{bethe}
Particle Data Group, 33. Passage of particles through matter 1,
\url{https://pdg.lbl.gov/2019/reviews/rpp2018-rev-passage-particles-matter.pdf}
\bibitem{typeinversion}
M. Moll, “Displacement Damage in Silicon Detectors for High Energy Physics”, IEEE Transactions on Nuclear Science, Aug. 2018. DOI: 10. 1109/TNS.2018.2819506.
\bibitem{timewalk}
Beniamino Di Girolamo, "The ATLAS Pixel Detector" (2011), \url{https://pos.sissa.it/137/006/pdf}
\bibitem{rd53a}
Garcia-Sciveres, Maurice, “The RD53A Integrated Circuit” (2019), \url{https://cds.cern.ch/record/2287593}, CERN Document Server, Modified Sep 2019.
\bibitem{calibsoft}
Maria Elena Stramaglia, "Calibration analysis software for the ATLAS Pixel Detector" (2015), \url{https://cds.cern.ch/record/2027805/files/ATL-INDET-PROC-2015-003.pdf}
\bibitem{pixeldb}
M. Verducci, "ATLAS conditions database experience with the LCG COOL conditions database project" (2008),
\url{https://iopscience.iop.org/article/10.1088/1742-6596/119/4/042031}
\bibitem{tothennka}
ATLAS physics plots, \url{https://atlas.web.cern.ch/Atlas/GROUPS/PHYSICS/PLOTS/PIX-2018-009/fig_05.png}, 2021年12月閲覧
\bibitem{pixeltypes}
Daniel Adam Dobos, "Production accompanying testing of the ATLAS Pixel module" (2017),
\url{https://cds.cern.ch/record/1016933/files/thesis-2007-016.pdf}
\bibitem{lingxin}
Lingxin Meng, "RD53A Module Testing Document" (最終更新: 2021年9月), \\
\url{https://cds.cern.ch/record/2702738/files/ATL-COM-ITK-2019-045.pdf}
\bibitem{electricaldoc}
Massimiliano Antonello, "Digital module electrical testing" (2020), \\
\url{https://cds.cern.ch/record/2723333/files/ATL-COM-ITK-2020-020.pdf}
\bibitem{sldo}
Vasilije Perovic, CMS Tracker Group. “Serial powering in four-chip prototype RD53A modules for Phase 2 upgrade of the CMS pixel detector”. ScienceDirect. 2020-10. \url{https://www.sciencedirect.com/science/article/pii/S0168900220308330}
\bibitem{mongo}
MongoDB, "MongoDB Documentation", \url{https://docs.mongodb.com}, 2021年12月閲覧
\bibitem{human}
Neville W. Sachs, P.E., "Practical Plant Failure Analysis" (2019), Economics, Finance, Business \& Industry, Engineering \& Technology
\end{thebibliography}

%
%
\chapter*{謝辞}
\addcontentsline{toc}{chapter}{謝辞}

shaji




%
\listoffigures
%
\listoftables
%
%
\newpage
\printindex
%
%
\end{document}
