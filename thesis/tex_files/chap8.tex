%------------------------------------------------------------------------------------------------------------------------
\chapter{まとめ}
\label{sec:chap8}
%------------------------------------------------------------------------------------------------------------------------

%------------------------------------------------------------------------------------------------------------------------
\section{結論}
\label{sec:ketsuron}
%------------------------------------------------------------------------------------------------------------------------

CERNにある世界最高エネルギーでの陽子衝突加速器LHC上の測定点の1つであるATLAS実験では、標準模型の精密測定や標準模型を超える物理現象の探索が行われている。LHCは2022年の3月に長期運転停止期間を終えてRun3として稼働を再開する。
ATLASピクセル検出器はこれまで2週間から1か月に1度の電荷較正を行っていた。電荷較正を行った後にそれぞれのパラメータが適切な値を保持していることを確認および欠陥しているデータの補完作業を行った後に、データベースにパラメータの登録を行う。データの確認および補完作業はこれまで担当者による手作業で行われていた。しかし、Run1から稼働しているピクセル検出器はこれまでよりも放射線損傷による影響がより大きくなることから、より頻繁に電荷較正およびその補完作業を行う必要があり、手作業による確認および補完作業は非常に労力が伴う。さらに、手作業による補正であることから、担当者によって補完方法の偏りが生じてしまう。そのため、電荷構成結果の適切な補完手法を確立し、自動で補完を行う解析ツールが必要である。

本研究では、電荷較正結果を確認し補完を行う解析ツールの開発を行った。電荷較正の際に発生しうる問題は2種類ある。1つ目の問題は、電荷較正を行う際に正しい試験電荷が生成できないことがある。この問題を検知するために電荷較正結果の新たな評価方法を導入し、問題のある試験電荷を順に取り除くアルゴリズムを開発した。2つ目の問題は、電荷較正結果に含まれるパラメータの欠陥である。これまでの補完方法は最も近いFEチップから値をコピーするという方法であり、担当者により異なる値による補完を行ってしまうことがあるため、パラメータの最適な補完方法の評価を行った。その結果、Threshold値については同一FEチップにおける異なるピクセルタイプの平均、その他のパラメータについては異なるFEチップにおける同一ピクセルタイプの平均を用いることにより、より実際の値に近い値を再現できるという結果が得られた。この結果を利用し、電荷較正結果に含まれるパラメータの欠陥を自動補完する解析ツールの開発を行った。開発した解析ツールを用いて2022年9月に行われた電荷較正データを用いて、Run3モンテカルロシミュレーションサンプル作成のための電荷較正結果の作成を行った。

また、統計数を増加させ新物理発見の感度を向上させるため、LHCでは2024年からHL-LHCへのアップグレードが計画されている。HL-LHCでは、陽子ビームのバンチに含まれる陽子数が増加するため、瞬間ルミノシティが現行LHCの$5$-$7$倍になり取得統計量の向上が期待される。そのため、1バンチあたりの信号数が増加するため、検出器には読み出し速度の高速化と、高い放射線耐性、およびイベントのパイルアップを防ぐために高細密化が要求される。陽子の衝突点から最も近い内部飛跡検出器\footnote{\ref{sec:InnerDetector}節で説明したように、内部飛跡検出器はIBL、ピクセル検出器、ストリップ検出器、遷移放射検出器から構成される。}は、要求性能を満たすために検出器の総入れ替えが予定されている。そのために、次世代ピクセルモジュールの大量生産が予定されており、各ピクセルモジュールに対して品質試験を行う。さらに、ATLASに搭載する際のモジュール選別や運転前後の性能比較のために、品質試験結果はチェコにある中央データベースに保管しておく必要がある。

本研究では、効率の良い量産と統合されたモジュール選定のために、品質試験結果の表示機能、品質試験結果の管理機能、および中央データベースとの同期機能の開発を行った。これらの機能をローカルデータベースに実装することにより、ピクセルモジュールの次世代器量産における品質試験結果管理に必要な機能の基本的な部分が全て揃った。しかし、品質試験結果を中央データベースへアップロードする機能について、読み出し試験結果のアップロード処理に約$160$秒必要という結果が得られた。処理時間改善のため、読み出し試験結果のアップロード処置を細分化し、処理時間を調査した。その結果、ローカルデータベースから試験結果を抽出処理を並列化することにより、処理時間が最大$50$秒程度削減できるという結果が得られた。


%------------------------------------------------------------------------------------------------------------------------
\section{今後の課題}
\label{sec:konngonokadai}
%------------------------------------------------------------------------------------------------------------------------

%------------------------------------------------------------------------------------------------------------------------
\subsection{電荷較正の自動補完ツール}
\label{sec:dennkahoseinokonngonokadai}
%------------------------------------------------------------------------------------------------------------------------

Run3においてピクセル検出器の電荷較正は10日に1度程度の頻度で行われる。この作業は1人の担当者により行われる予定であるが、担当者が体調不良等が原因で作業を行うことができなくなると別の人が作業を行うことになる。本研究において開発した電荷較正結果の補完ツールについて、補完方法の説明や使用方法の説明のためのドキュメント作成を行う必要がある。

また、2021年9月に行われた電荷較正のデータを用いて電荷較正の補完を行った。その際に、電荷較正結果を確認すると、小さい試験電荷を生成した際に2つの構造が確認でき、ToT$=4$が得られた際に、較正式から得られる電荷と試験電荷の違いが20\% 程度あるということがわかった。このような違いがクラスタリングから得られる荷電粒子の通過位置に、どのような影響を及ぼすかをシミュレーション等を用いて正確に評価する必要がある。

%補完ツールはコマンドラインベース(CUI)にて実行するものである

%------------------------------------------------------------------------------------------------------------------------
\subsection{次世代ピクセルモジュール量産のためのデータベースシステム}
\label{sec:dbnokonngonokadai}
%------------------------------------------------------------------------------------------------------------------------

本研究では、読み出し試験以外の項目についての閲覧機能や同期機能等を開発することにより、次世代ピクセルモジュールの量産に必要な機能の基本的な部分が揃った。ローカルデータベースシステムは各組み立て機関に設置されるものであるため、国外の研究機関が使用できるようにユーザーサポートを行う必要がある。そのために、ソフトウェア使用のためのドキュメントの作成やグループチャットによる問合せ対応を行っている。このようなユーザーサポートを今後も継続していく必要がある。

また、前章で述べたように、ローカルデータベースから中央データベースへの品質試験結果のアップロード機能のように、処理時間が長くユーザーの使用満足度が非常に悪い部分が確認された。読み出し試験結果の閲覧機能についても、結果表示のために数分必要ということが確認されていることから、処理時間を短くするような工夫が必要である。

\newpage
