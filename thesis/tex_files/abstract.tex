\chapter*{Abstract}





\newpage

\chapter*{概要}

フランスとスイスの国境にある欧州原子力研究機構(CERN)に建設された陽子衝突加速器LHCでは、世界最高エネルギーである重心系エネルギー$14\ \si{TeV}$で陽子同士を衝突させることができる。LHCの衝突点の1つに設置されているATLASでは、陽子衝突からの崩壊生成粒子を測定し、標準模型の精密測定やそれを超える新物理の探索を行っている。

ATLASの最内層に設置されているシリコンピクセル検出器は、放射線損傷による影響を補正するために頻繁に電荷較正を行う必要がある。電荷較正結果には、ASICが正しい電荷を生成できないことや、データの欠陥等の問題が含まれることがあるため、これら値を適切な値に補完する必要がある。本研究では、電荷較正結果により適切な例外処理を行うよう、例外をアルゴリズムとして抽出・処理する解析ツールを開発した。
開発した解析ツール用いて、2021年9月に取得した最新の電荷較正データに対して電荷較正およびその補正を行った。最新の電荷較正データを用いて作成した結果から、小さい電荷量について電荷較正が理想的に行えていないことがわかった。この電荷較正結果が小さい電荷量を含む測定データに与える影響を評価を行った。その結果、ATLASピクセル検出器における主な測定対象であるMIP粒子に対して、小さい電荷量を含むクラスターへの影響は小さいと予想され、電荷較正式から得られる電荷量の補正を行わないことに決定した。

また、LHC高輝度化(HL-LHC)に向けたATLAS検出器アップグレードのため、次世代ピクセル検出器の開発および量産の準備を進めている。HL-LHCでは、瞬間ルミノシティが現在のLHCの$5$-$7$倍、積分ルミノシティが$2025$年までの約$10$倍になり、標準模型の測定精度の向上や新物理の探索感度の向上が期待される。瞬間ルミノシティの増加に伴い、データレートの増加やヒット占有率の増加することにより、検出器に対してより高い性能が要求される。さらに、現行ATLASピクセル検出器は2025年における稼動後、設計放射線耐性に到達するため、再内層に設置されている内部飛跡検出器の総入れ替えを予定しており、新しく設置する検出器をITk (Inner Tracker)と呼ぶ。

ITkの製造のために、次世代ピクセルモジュール約10000台を量産し、全てのモジュールに対して品質試験を行う予定となっている。効率の良い量産と統合されたピクセルモジュール選定を行うために、品質試験結果を統合管理するシステムが必要となる。そのために、東工大を中心としてローカルデータベースという統合管理システムの開発が進められている。
本研究では、先行研究において開発された読み出し試験についての管理機能に加え、外観鑑別や形状測定などの試験項目についての管理機能、モジュール登録機能、試験結果の共有機能の開発を行った。

以上の機能を新たに実装することにより、ピクセルモジュールの次世代器量産における品質試験結果管理に必要な機能の基本的な部分が全て揃った。しかし、一部の機能において非常に長い処理時間を必要とすることが確認された。このような長い処理時間は使用者の満足度が非常に悪くなるため、改善が必要である。そのため、処理内容を細分化することにより改善方法を指摘し、今後の課題とした。


%世界最高エネルギーでの陽子衝突加速器LHCで新物理の発見を目指すATLAS実験のピクセル検出器においては、電荷較正の結果にデータの欠損や較正の失敗が含まれると、実測およびシミュレーションに影響を及ぼすため、較正結果を評価し再較正を行う必要がある。本研究では、再較正の際により適切な欠損の補完処理を行うよう、例外をアルゴリズムとして抽出・処理する自動解析ツールを開発した。
%
%また、LHC高輝度化に向けたATLAS検出器アップグレードのため、新型ピクセル検出器の開発および量産の準備を進めている。検出器の品質管理のために、組立工程において様々な試験を行う。本研究では、効率の良い量産と統合されたモジュール選定のために、先行する読み出し試験についての管理機能に加え、外観鑑別や形状測定などの試験項目についての管理機能、モジュール登録機能、試験結果の共有機能の開発を行った。
