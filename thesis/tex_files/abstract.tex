\chapter*{Abstract}



\newpage

\chapter*{概要}

%世界最高エネルギーでの陽子衝突加速器LHCで新物理の発見を目指すATLAS実験のピクセル検出器においては、電荷較正の結果にデータの欠損や較正の失敗が含まれると、実測およびシミュレーションに影響を及ぼすため、較正結果を評価し再較正を行う必要がある。本研究では、再較正の際により適切な欠損の補完処理を行うよう、例外をアルゴリズムとして抽出・処理する自動解析ツールを開発した。

%また、LHC高輝度化に向けたATLAS検出器アップグレードのため、新型ピクセル検出器の開発および量産の準備を進めている。検出器の品質管理のために、組立工程において様々な試験を行う。本研究では、効率の良い量産と統合されたモジュール選定のために、先行する読み出し試験についての管理機能に加え、外観鑑別や形状測定などの試験項目についての管理機能、モジュール登録機能、試験結果の共有機能の開発を行った。
