\chapter{序論}

フランスとスイスの国境にある欧州原子力研究機構 (CERN) に設置されている大型陽子衝突型加速器 (LHC) では、現在、素粒子物理学の基礎となっている標準模型の精密測定や標準模型を超える物理現象の探索が行われている。 ATLAS実験は HC上にある4つの衝突点の1つで行われている実験であり、ATLAS 検出器を用いて 生成粒子の測定が行われている。LHCでは加速器のアップグレード(HL-LHC)を予定しており、これに向けてATLAS検出器のアップグレードを行う。この章ではLHC-ATLAS実験とそのアップグレード計画について説明する。


\section{素粒子標準模型}






\section{LHC}

\subsection{LHCの基本構造}






\section{ATLAS実験}
\begin{figure}[tbp]
  \centering
  \includegraphics[height=7cm,keepaspectratio]{ATLAS.jpg}
  \caption[ATLAS検出器]{ATLAS検出器の全体図 \cite{ATLAS} }
  \label{fig:ATLAS}
\end{figure}


ATLAS(\textbf{A} \textbf{T}roidala \textbf{L}HC \textbf{A}pparata\textbf{S})実験はLHCの衝突点の一つに設置されている汎用型の検出器である。\fref{fig:ATLAS} に示すように、ATLAS検出器は直径25\ \si{m}長さ44\ \si{m}の円筒型をした巨大な検出器である。その中心に陽子の衝突点があり、LHCによって加速された陽子ビームが円筒の中心軸を通過するような構造になっている。
陽子ビームの衝突点である円筒の中心の内側から順に、内部飛跡検出器、電磁カロリメータ、ハドロンカロリメータ、ミューオン検出器が衝突点を覆うように存在する。内部飛跡検出器と電磁カロリメータの間にはソレノイド磁石、ハドロンカロリメータの外側にはトロイド磁石が


\subsection{内部秘蹟検出器}

\subsection{カロリメータ}

\subsection{ミューオン検出器}









\section{HL-LHCアップグレード}



\newpage
